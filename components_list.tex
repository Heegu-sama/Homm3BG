% !TeX spellcheck = en_US
\documentclass[12pt]{article}

% Language setting
\usepackage[english]{babel}

% Set page size and margins
\usepackage[
  a4paper,
  top=2cm,
  bottom=3cm,
  left=2cm,
  right=2cm,
  marginparwidth=1.75cm,
  footskip=2.05cm
]{geometry}

% Useful packages
\usepackage[export]{adjustbox}
\usepackage{amsmath}
\usepackage{caption}
\usepackage[strict]{changepage}
\usepackage{enumitem}
\usepackage{float}
\usepackage{fullwidth}
\usepackage{graphicx, trimclip}
\usepackage[colorlinks=true, allcolors=blue]{hyperref}
\usepackage{hyperref}
\usepackage[nonewpage]{imakeidx}
\usepackage{multicol}
\usepackage{outlines}
\usepackage{setspace}
\usepackage{stfloats}
\usepackage{subfigure}
\usepackage[usetransparent=false]{svg}
\usepackage[subfigure]{tocloft}
\usepackage{tikz}
\usepackage{titlesec}
\usepackage{varwidth}
\usepackage{wrapfig}
\usepackage[most]{tcolorbox}
\newtcolorbox{scaledfigure}[1][]{height fill, space to=\myspace,#1}
\hypersetup{
  colorlinks=true,
  linkcolor=goldenbrown,
  filecolor=magenta,
  urlcolor=cyan,
  pdftitle={Heroes of Might \& Magic III Rule Book},
  pdfpagemode=FullScreen,
}
% Set the default spacing between paragraphs. Remove indentation.
\usepackage[skip=6pt, indent=0pt]{parskip}
\setstretch{1}

\setlength{\columnsep}{1cm}

% Variables
\def\_assets{assets}

\def\art{\_assets/art}
\def\cards{\_assets/cards}
\def\examples{\_assets/examples}
\def\images{\_assets/images}
\def\layout{\_assets/layout}
\def\map_locations{\_assets/map-locations}
\def\skills{\_assets/skills}
\def\spells{\_assets/spells}
\def\svgs{\_assets/glyphs}
\def\notes_svgs{\svgs/for-notes}
\def\tables{\_assets/tables}
\def\qr{\_assets/qr-codes}

% Colors
\definecolor{darkcandyapplered}{rgb}{0.64, 0.0, 0.0}
\definecolor{antiquewhite}{rgb}{0.98, 0.92, 0.84}
\definecolor{goldenbrown}{rgb}{0.6, 0.4, 0.08}
\definecolor{arylideyellow}{rgb}{0.91, 0.84, 0.42}
\definecolor{amber}{rgb}{1.0, 0.49, 0.0}


\titleformat{\section}
{\huge}
{\filright
\footnotesize
\enspace SECTION \thesection\enspace}
{8pt}
{\Huge\bfseries\filcenter\uppercase}
%Create section heading with graphics. Argument one is heading name, argument two is picture to use on the left.
\newcommand{\addsection}[2]{
  \vspace*{-5em}
  \hspace*{-1em}
  \makebox[0pt][l]{
  \raisebox{-\totalheight}[0pt][7pt]{
    \begin{tikzpicture}
      \draw (0, 0) node[inner sep=0] {\includegraphics[width=\linewidth, height=0.2\linewidth]{\layout/section_heading.jpg}};
      \draw (-6.2, 0) node {\includegraphics[width=0.125\textwidth]{#2}};
    \end{tikzpicture}
    }
  }
  \begin{fullwidth}[leftmargin=0.16\textwidth]
    \begin{center}
      \fontfamily{ptm}\selectfont{
        \color{antiquewhite} \section*{#1}
        \cleardoublepage\phantomsection\addcontentsline{toc}{section}{\protect\numberline{}#1}
      }
    \end{center}
  \end{fullwidth}
  \vspace{1.75em}
}
%End of create section heading.


% Background
\AddToHook{shipout/background}{%
  \put (0in,-\paperheight){\includegraphics[width=\paperwidth,height=\paperheight]{\layout/tausta.png}}%
  \put (0in,-\paperheight){\includegraphics[width=\paperwidth,height=0.05\paperheight]{\layout/bottom.png}}%
}

\begin{document}

\addsection{Inferno}{\covers/inferno.jpg}

\begin{multicols}{3}

\small

\begin{itemize}[leftmargin=0pt, label={}, noitemsep, noitemsep]
  \item \textbf{\underline{Books Leaflets}} (2)
  \item Player's Aid
  \item Inferno Mission Book
\end{itemize}

\begin{itemize}[leftmargin=0pt, label={}, noitemsep, noitemsep]
  \item \textbf{\underline{Hero Cards}} (2)
  \item Fiona/Xyron
  \item Rashka/Zydar
\end{itemize}

\begin{itemize}[leftmargin=0pt, label={}, noitemsep, noitemsep]
  \item \textbf{\underline{Town Boards}} (1)
  \item Inferno
\end{itemize}

\textbf{\underline{Map Tiles}} (7)\\
C5, F16, F17, F18, N11, N12, S6

\begin{itemize}[leftmargin=0pt, label={}, noitemsep, noitemsep]
  \item \textbf{\underline{Miscellaneous Tokens}} (29)
  \item 1 Gold (3)
  \item 3 Gold (3)
  \item 10 Gold (3)
  \item 1 Building material (3)
  \item 3 Building material (4)
  \item 1 Valuable (3)
  \item 3 Valuable (2)
  \item 1/2 damage (2)
  \item 3/5 damage (2)
  \item Build tokens (1) {red}
  \item Population tokens (1) {red}
  \item Spell Book (1) {red}
  \item Morale (1)
\end{itemize}

\begin{itemize}[leftmargin=0pt, label={}, noitemsep, noitemsep]
  \item \textbf{\underline{Acrylic Cubes}} (30)
  \item Red (20)
  \item Black (10)
\end{itemize}

\begin{itemize}[leftmargin=0pt, label={}, noitemsep, noitemsep]
  \item \textbf{\underline{Miniatures}} (10)
  \item Inferno Heroes (2 poses)
  \item Inferno town (1)
  \item Familiars (1)
  \item Magogs (1)
  \item Cerberi (1)
  \item Demons (1)
  \item Pit lords (1)
  \item Efreet (1)
  \item Arch devils (1)
\end{itemize}

\begin{itemize}[leftmargin=0pt, label={}, noitemsep, noitemsep]
  \item \textbf{\underline{Cards}} (64)
  \item
  \item \underline{Unit Cards} (7)
  \item Inferno (7)
  \item Familiars (1)
  \item Magogs (1)
  \item Cerberi (1)
  \item Demons (1)
  \item Pit lords (1)
  \item Efreet (1)
  \item Arch devils (1)
  \item
  \item \underline{Neutral Unit Cards} (7)
  \item \underline{Bronze} (3)
  \item Familiars (1)
  \item Magogs (1)
  \item Cerberi
  \item \underline{Silver} (2)
  \item Demons (1)
  \item Pit lords (1)
  \item \underline{Gold} (2)
  \item Efreet (1)
  \item Arch devils (1)
  \item
  \item \underline{Statistic Cards} (7)
  \item Attack (2)
  \item Defense (2)
  \item Knowledge (1)
  \item Power (2)
  \item
  \item \underline{Empowered Statistic Cards} (20)
  \item Attack (5)
  \item Defense (5)
  \item Knowledge (5)
  \item Power (5)
  \item
  \item \underline{Astrologers Proclaim} (3)
  \item Dancing Imp (1)
  \item Explorers (1)
  \item Hero (1)
  \item
  \item \underline{Spells} (6)
  \item Inferno (2)
  \item Magic Arrow (2)
  \item Visions (2)
  \item
  \item \underline{Hero Specialties} (12)
  \item \underline{Fiona} (3)
  \item Cerberi I (1)
  \item Cerberi IV (1)
  \item Cerberi VI (1)
  \item \underline{Xyron} (3)
  \item Inferno I (1)
  \item Inferno IV (1)
  \item Inferno VI (1)
  \item \underline{Rashka} (3)
  \item Efreet I (1)
  \item Efreet IV (1)
  \item Efreet VI (1)
  \item \underline{Zydar} (3)
  \item Sorcery I (1)
  \item Sorcery IV (1)
  \item Sorcery VI (1)
  \item
  \item \underline{Artifacts} (4)
  \item Boots of Speed (1)
  \item Breastplate of brimstone (1)
  \item Crown of dragontooth (1)
  \item Shield of the damned (1)
  \item
  \item \underline{Abilities} (5)
  \item Ballistics (2)
  \item Scholar (2)
  \item Scouting (1)
  \item
  \item \underline{Other} (1)
  \item Inferno deck cards (1)
\end{itemize}

\end{multicols}


\addsection{Battlefield}{\covers/battlefield.jpg}

\begin{multicols}{3}

\footnotesize

\begin{itemize}[leftmargin=0pt, label={}, noitemsep]
  \item \textbf{\small{\underline{Books Leaflets}}} (3)
  \item
  \item Battlefield Rulebook (1)
  \item Player's Aid Battlefield (2)
\end{itemize}

\begin{itemize}[leftmargin=0pt, label={}, noitemsep]
  \item \textbf{\small{\underline{Boards}}} (1)
  \item
  \item Battlefield (1)
\end{itemize}

\begin{itemize}[leftmargin=0pt, label={}, noitemsep]
  \item \textbf{\small{\underline{Obstacle Tokens}}} (10)
  \item
  \item Fire (2)
  \item Lake/Lake (1)
  \item Log/Castle Wall (2)
  \item Rock/Castle Wall (1)
  \item Skeleton, Stump/Castle Gate (1)
  \item Skeleton/Skeleton (1)
  \item Skull, Rock/Skull, Rock (1)
  \item Skull/Castle Wall (1)
\end{itemize}

\textbf{\small{\underline{Initiative Tokens}}} (1)

\vspace*{\fill}
\columnbreak

\begin{itemize}[leftmargin=0pt, label={}, noitemsep]
  \item \textbf{\small{\underline{Cards}}} (70)
  \item
  \item \textbf{Adventure Cards} (50)
  \item Campfire (2)
  \item Corpse (2)
  \item Crypt (2)
  \item Cyclops Stockpile (2)
  \item Dragon Utopia (2)
  \item Dwarven Treasury (2)
  \item Imp Cache (2)
  \item Learning Stone (2)
  \item Magic Spring (2)
  \item Obelisk (4)
  \item Pandora's Box (2)
  \item Scholar (2)
  \item Shrine of Magic Gesture (2)
  \item Spell Scroll (2)
  \item Temple (2)
  \item Trading Post (6)
  \item Treasure Chest (2)
  \item Tree of Knowledge (2)
  \item Warrior's Tomb (2)
  \item Water Wheel (2)
  \item Windmill (2)
  \item Witch Hut (2)
\columnbreak
  \item \textbf{Negative Morale Cards} (10)
  \item Attack set one to -1 (1)
  \item Discard 1 card at random (2)
  \item Paralyze one unit (1)
  \item Reroll +1 on attack die (2)
  \item Roll one less die when doing 2 search or 2 treasure (1)
  \item Search(x) is search(1) (1)
  \item Skip unit activation on -1 attack die (1)
  \item Suffer -1 to next attack, defense or power roll (1)
  \item
  \item \textbf{Positive Morale Cards} (10)
  \item Remove paralyze one unit (1)
  \item +1 attack, +1 defense or +1 power on next combat (1)
  \item Discard an adventurer card and draw another (1)
  \item Discard any number of cards and draw as many (1)
  \item Do another search(x) (1)
  \item Draw a card on combat start (2)
  \item Next attack set a die to +1 (1)
  \item Reroll a die (2)
  \item
  \item \textbf{Other Cards} (1)
  \item Battlefield deck cards (1)
\end{itemize}

\end{multicols}

\vfill
\begin{center}
  \includegraphics[width=0.9\linewidth]{\images/battlefield.png}
\end{center}


\end{document}
