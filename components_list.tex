% !TeX spellcheck = en_US
\documentclass[12pt]{article}

% Language setting
\usepackage[english]{babel}

% Set page size and margins
\usepackage[
  a4paper,
  top=2cm,
  bottom=3cm,
  left=2cm,
  right=2cm,
  marginparwidth=1.75cm,
  footskip=2.05cm
]{geometry}

% Useful packages
\usepackage[export]{adjustbox}
\usepackage{amsmath}
\usepackage{caption}
\usepackage[strict]{changepage}
\usepackage{enumitem}
\usepackage{float}
\usepackage{fullwidth}
\usepackage{graphicx, trimclip}
\usepackage[colorlinks=true, allcolors=blue]{hyperref}
\usepackage{hyperref}
\usepackage[nonewpage]{imakeidx}
\usepackage{multicol}
\usepackage{outlines}
\usepackage{setspace}
\usepackage{stfloats}
\usepackage{subfigure}
\usepackage[usetransparent=false]{svg}
\usepackage[subfigure]{tocloft}
\usepackage{tikz}
\usepackage{titlesec}
\usepackage{varwidth}
\usepackage{wrapfig}
\usepackage[most]{tcolorbox}
\newtcolorbox{scaledfigure}[1][]{height fill, space to=\myspace,#1}
\hypersetup{
  colorlinks=true,
  linkcolor=goldenbrown,
  filecolor=magenta,
  urlcolor=cyan,
  pdftitle={Heroes of Might \& Magic III Components List},
  pdfpagemode=UseNone,
}
% Set the default spacing between paragraphs. Remove indentation.
\usepackage[skip=6pt, indent=0pt]{parskip}
\setstretch{1}

% Add dots to the table of contents
\renewcommand{\cftsecleader}{\cftdotfill{\cftsecdotsep}}
\renewcommand\cftsecdotsep{\cftdot}
\renewcommand\cftsubsecdotsep{\cftdot}

\captionsetup[figure]{labelformat=empty}
\usetikzlibrary{shadows, shadows.blur, calc}

\setlength{\columnsep}{1cm}

% Variables
\def\_assets{assets}

\def\art{\_assets/art}
\def\cards{\_assets/cards}
\def\examples{\_assets/examples}
\def\images{\_assets/images}
\def\layout{\_assets/layout}
\def\map_locations{\_assets/map-locations}
\def\skills{\_assets/skills}
\def\spells{\_assets/spells}
\def\svgs{\_assets/glyphs}
\def\notes_svgs{\svgs/for-notes}
\def\tables{\_assets/tables}
\def\qr{\_assets/qr-codes}
\def\covers{\_assets/box-covers}

% Colors
\definecolor{darkcandyapplered}{rgb}{0.64, 0.0, 0.0}
\definecolor{antiquewhite}{rgb}{0.98, 0.92, 0.84}
\definecolor{goldenbrown}{rgb}{0.6, 0.4, 0.08}
\definecolor{arylideyellow}{rgb}{0.91, 0.84, 0.42}
\definecolor{amber}{rgb}{1.0, 0.49, 0.0}

% Command to frame images
\newcommand\framedimage[2][]{%
  \begin{tikzpicture}
    \draw (0, 0) node[inner sep=0] {\makebox[#1][c]{\includegraphics[width=#1]{#2}}};
    \draw [borderoutyellow, thick] (current bounding box.north west) rectangle (current bounding box.south east);
  \end{tikzpicture}}
% End of drop frame definition

\titleformat{\section}
{\huge}
{\filright
\footnotesize
\enspace SECTION \thesection\enspace}
{8pt}
{\Huge\bfseries\filcenter\uppercase}
%Create section heading with graphics. Argument one is heading name, argument two is picture to use on the left.
\newcommand{\addsection}[2]{
  \vspace*{-5em}
  \hspace*{-1em}
  \makebox[0pt][l]{
  \raisebox{-\totalheight}[0pt][7pt]{
    \begin{tikzpicture}
      \draw (0, 0) node[inner sep=0] {\includegraphics[width=\linewidth, height=0.2\linewidth]{\layout/section_heading.jpg}};
      \draw (-6.2, 0) node {\framedimage[0.14\textwidth]{#2}};
    \end{tikzpicture}
    }
  }
  \begin{fullwidth}[leftmargin=0.16\textwidth]
    \begin{center}
      \fontfamily{ptm}\selectfont{
        \color{antiquewhite} \section*{#1}
        \cleardoublepage\phantomsection\addcontentsline{toc}{section}{\protect\numberline{}#1}
      }
    \end{center}
  \end{fullwidth}
  \vspace{1.75em}
}
%End of create section heading.

\newcommand\picdims[4][]{%
  \setbox0=\hbox{\includegraphics[#1]{#4}}%
  \clipbox{.5\dimexpr\wd0-#2\relax{} %
    .5\dimexpr\ht0-#3\relax{} %
    .5\dimexpr\wd0-#2\relax{} %
    .5\dimexpr\ht0-#3\relax}{\includegraphics[#1]{#4}}}

\tikzset{
  thick/.style=      {line width=1.3pt},
  very thick/.style= {line width=1.7pt},
  ultra thick/.style={line width=2.2pt}
}

\definecolor{borderoutyellow}{HTML}{E0B75F}
\definecolor{borderinyellow}{HTML}{F6EA48}
% Create note box
\newcommand{\note}[2]{
  \begin{tikzpicture}
    \draw (0, 0) node[inner sep=0] {\makebox[\linewidth][c]{\picdims[width=\linewidth]{\linewidth}{#1\baselineskip}{\layout/note-big.jpg}}};
    \draw [black, ultra thick] ([xshift=+2pt, yshift=-2pt] current bounding box.north west) rectangle ([xshift=-2pt, yshift=2pt] current bounding box.south east);
    \draw [borderoutyellow, very thick] (current bounding box.north west) rectangle (current bounding box.south east);
    \draw [borderinyellow, thick] ([xshift=+4.5pt, yshift=-4.5pt] current bounding box.north west) rectangle ([xshift=-4.5pt, yshift=4.5pt] current bounding box.south east);
    \node at (current bounding box.center) {
      \begin{varwidth}{0.85\linewidth}
      \fontfamily{ptm}\selectfont{
        \color{arylideyellow}
        \hypersetup{linkcolor=amber}
        #2
        \hypersetup{linkcolor=goldenbrown}
      }
      \end{varwidth}
    };
  \end{tikzpicture}
}

% Command for overlay circled text
\definecolor{goblin}{HTML}{3b7c33}
\newcommand\encircle[1]{%
  \tikz[baseline=(X.base)]
  \node (X) [draw=white, shape=circle, inner sep=0, fill=goblin, text=white, blur shadow={shadow blur steps=5}] {\strut \textbf{#1}};%
}

% Background
\AddToHook{shipout/background}{%
  \put (0in,-\paperheight){\includegraphics[width=\paperwidth,height=\paperheight]{\layout/tausta.png}}%
  \put (0in,-\paperheight){\includegraphics[width=\paperwidth,height=0.05\paperheight]{\layout/bottom.png}}%
}

\makeindex[columns=3, title=, options={-s index_style.ist}]

\title{\includegraphics[width=6cm]{\images/title.png}\\Components List}

\begin{document}

\maketitle

\begin{center}
  Version 0.4

  \bigbreak
  This document aims to list comprehensively the contents of all the board game boxes.
  It was compiled using a list in \href{https://boardgamegeek.com/thread/3265461/article/43995671#43995671}{this BoardGameGeek thread}.
  \par
  It is a community-driven project, which has a \href{https://github.com/Heegu-sama/Homm3BG}{GitHub repository}.
  Everyone is welcome to contribute, make changes, and fix errors.
\end{center}

\bigbreak

\tableofcontents

\addsection{Core Game}{\covers/core_game.jpg}


\addsection{Rampart}{\covers/rampart.jpg}


\addsection{Fortress}{\covers/fortress.jpg}

\begin{multicols*}{3}

\footnotesize

\begin{itemize}[leftmargin=0pt, label={}, noitemsep]
  \item \textbf{\small{\underline{Books Leaflets}}} (2)
  \item
  \item Player's Aid (1)
  \item Fortress Mission Book (1)
\end{itemize}

\begin{itemize}[leftmargin=0pt, label={}, noitemsep]
  \item \textbf{\small{\underline{Hero Cards}}} (1)
  \item
  \item Wystan/Adrienne (1)
\end{itemize}

\begin{itemize}[leftmargin=0pt, label={}, noitemsep]
  \item \textbf{\small{\underline{Town Boards}}} (1)
  \item
  \item Fortress (1)
\end{itemize}

\textbf{\small{\underline{Map Tiles}}} (7)\\

C4, F13, F14, F15, N10, N9, S5

\begin{itemize}[leftmargin=0pt, label={}, noitemsep]
  \item \textbf{\small{\underline{Miscellaneous Tokens}}} (23)
  \item
  \item 1 Gold (3)
  \item 3 Gold (3)
  \item 10 Gold (3)
  \item 1 Building material (3)
  \item 3 Building material (3)
  \item 1 Valuable (3)
  \item 3 Valuable (1)
  \item Build tokens (1)
  \item Population tokens (1)
  \item Spell Book (1)
  \item Morale (1)
\end{itemize}

\begin{itemize}[leftmargin=0pt, label={}, noitemsep]
  \item \textbf{\small{\underline{Acrylic Cubes}}} (30)
  \item
  \item Dark Green (20)
  \item Black (10)
\end{itemize}

\begin{itemize}[leftmargin=0pt, label={}, noitemsep]
  \item \textbf{\small{\underline{Miniatures}}} (9)
  \item
  \item Fortress Heroes (2 poses)
  \item Gnolls (1)
  \item Lizardmen (1)
  \item Dragon flies (1)
  \item Basilisks (1)
  \item Gorgons (1)
  \item Wyverns (1)
  \item Hydras (1)
\end{itemize}

\columnbreak

\begin{itemize}[leftmargin=0pt, label={}, noitemsep]
  \item \textbf{\small{\underline{Cards}}} (78)
  \item
  \item \textbf{Unit Cards} (7)
  \item
  \item \textbf{Fortress} (7)
  \item Gnolls (1)
  \item Lizardmen (1)
  \item Dragon flies (1)
  \item Basilisks (1)
  \item Gorgons (1)
  \item Wyverns (1)
  \item Hydras (1)
  \item
  \item \textbf{Neutral Unit Cards} (2)
  \item
  \item \textbf{Azure} (2)
  \item Rust dragons (2)
  \item
  \item \textbf{Statistics Cards} (7)
  \item Defense (4)
  \item Knowledge (2)
  \item Power (2)
  \item
  \item \textbf{Astrologers Proclaim} (3)
  \item Big Cleanup (1)
  \item Forty Thieves (1)
  \item Plane Between Planes (1)
  \item
  \item \textbf{Spells} (20)
  \item Fly (2)
  \item Fortune (2)
  \item Frenzy (2)
  \item Frost Ring (2)
  \item Implosion (2)
  \item Magic Arrow (4)
  \item Misfortune (2)
  \item Remove Obstacle (2)
  \item View Earth (2)
  \item
  \item \textbf{Hero Specialties} (6)
  \item
  \item \textbf{Wystan} (3)
  \item Lizardmen I (1)
  \item Lizardmen IV (1)
  \item Lizardmen VI (1)
  \item \textbf{Adrienne} (3)
  \item Fire Magic I (1)
  \item Fire Magic IV (1)
  \item Fire Magic VI (1)
  \item
  \item \textbf{Artifacts} (8)
  \item Arms of legion (1)
  \item Crest of valor (1)
  \item Endless purse of gold (1)
  \item Helm of heavenly enlightenment (1)
  \item Recanter's cloak (1)
  \item Scales of the greater basilisk (1)
  \item Spirit of oppression (1)
  \item Sword of hellfire (1)
  \item
  \item \textbf{Abilities} (4)
  \item Eagle eye (2)
  \item Learning (2)
  \item
  \item \textbf{Event Cards} (20)
  \item A Shady Auction (1)
  \item Artifact Merchant (1)
  \item Crypt (1)
  \item Cursed Swamp (1)
  \item Den of Thieves (1)
  \item Garden of Revelation (1)
  \item Library of Enlightenment (1)
  \item Mage Laboratory (1)
  \item Magical Forest (1)
  \item Market of Time (1)
  \item Marketplace (1)
  \item Mercenary Camp (1)
  \item Messenger with Supplies (1)
  \item Mischievous Leprechaun (1)
  \item Prison (1)
  \item School of Magic and School of War (1)
  \item Shrine of the Magic Thought (1)
  \item Stables (1)
  \item The Villager's Plea (1)
  \item Withered Hermit (1)
  \item
  \item \textbf{Other Cards} (1)
  \item Fortress Deck Cards (1)
\end{itemize}

\end{multicols*}


\addsection{Inferno}{\covers/inferno.jpg}

\begin{multicols}{3}

\small

\begin{itemize}[leftmargin=0pt, label={}, noitemsep, noitemsep]
  \item \textbf{\underline{Books Leaflets}} (2)
  \item Player's Aid
  \item Inferno Mission Book
\end{itemize}

\begin{itemize}[leftmargin=0pt, label={}, noitemsep, noitemsep]
  \item \textbf{\underline{Hero Cards}} (2)
  \item Fiona/Xyron
  \item Rashka/Zydar
\end{itemize}

\begin{itemize}[leftmargin=0pt, label={}, noitemsep, noitemsep]
  \item \textbf{\underline{Town Boards}} (1)
  \item Inferno
\end{itemize}

\textbf{\underline{Map Tiles}} (7)\\
C5, F16, F17, F18, N11, N12, S6

\begin{itemize}[leftmargin=0pt, label={}, noitemsep, noitemsep]
  \item \textbf{\underline{Miscellaneous Tokens}} (29)
  \item 1 Gold (3)
  \item 3 Gold (3)
  \item 10 Gold (3)
  \item 1 Building material (3)
  \item 3 Building material (4)
  \item 1 Valuable (3)
  \item 3 Valuable (2)
  \item 1/2 damage (2)
  \item 3/5 damage (2)
  \item Build tokens (1) {red}
  \item Population tokens (1) {red}
  \item Spell Book (1) {red}
  \item Morale (1)
\end{itemize}

\begin{itemize}[leftmargin=0pt, label={}, noitemsep, noitemsep]
  \item \textbf{\underline{Acrylic Cubes}} (30)
  \item Red (20)
  \item Black (10)
\end{itemize}

\begin{itemize}[leftmargin=0pt, label={}, noitemsep, noitemsep]
  \item \textbf{\underline{Miniatures}} (10)
  \item Inferno Heroes (2 poses)
  \item Inferno town (1)
  \item Familiars (1)
  \item Magogs (1)
  \item Cerberi (1)
  \item Demons (1)
  \item Pit lords (1)
  \item Efreet (1)
  \item Arch devils (1)
\end{itemize}

\begin{itemize}[leftmargin=0pt, label={}, noitemsep, noitemsep]
  \item \textbf{\underline{Cards}} (64)
  \item
  \item \underline{Unit Cards} (7)
  \item Inferno (7)
  \item Familiars (1)
  \item Magogs (1)
  \item Cerberi (1)
  \item Demons (1)
  \item Pit lords (1)
  \item Efreet (1)
  \item Arch devils (1)
  \item
  \item \underline{Neutral Unit Cards} (7)
  \item \underline{Bronze} (3)
  \item Familiars (1)
  \item Magogs (1)
  \item Cerberi
  \item \underline{Silver} (2)
  \item Demons (1)
  \item Pit lords (1)
  \item \underline{Gold} (2)
  \item Efreet (1)
  \item Arch devils (1)
  \item
  \item \underline{Statistic Cards} (7)
  \item Attack (2)
  \item Defense (2)
  \item Knowledge (1)
  \item Power (2)
  \item
  \item \underline{Empowered Statistic Cards} (20)
  \item Attack (5)
  \item Defense (5)
  \item Knowledge (5)
  \item Power (5)
  \item
  \item \underline{Astrologers Proclaim} (3)
  \item Dancing Imp (1)
  \item Explorers (1)
  \item Hero (1)
  \item
  \item \underline{Spells} (6)
  \item Inferno (2)
  \item Magic Arrow (2)
  \item Visions (2)
  \item
  \item \underline{Hero Specialties} (12)
  \item \underline{Fiona} (3)
  \item Cerberi I (1)
  \item Cerberi IV (1)
  \item Cerberi VI (1)
  \item \underline{Xyron} (3)
  \item Inferno I (1)
  \item Inferno IV (1)
  \item Inferno VI (1)
  \item \underline{Rashka} (3)
  \item Efreet I (1)
  \item Efreet IV (1)
  \item Efreet VI (1)
  \item \underline{Zydar} (3)
  \item Sorcery I (1)
  \item Sorcery IV (1)
  \item Sorcery VI (1)
  \item
  \item \underline{Artifacts} (4)
  \item Boots of Speed (1)
  \item Breastplate of brimstone (1)
  \item Crown of dragontooth (1)
  \item Shield of the damned (1)
  \item
  \item \underline{Abilities} (5)
  \item Ballistics (2)
  \item Scholar (2)
  \item Scouting (1)
  \item
  \item \underline{Other} (1)
  \item Inferno deck cards (1)
\end{itemize}

\end{multicols}


\addsection{Tower}{\covers/tower.jpg}


\addsection{Strech Goals Faction}{\covers/stretch_goals_faction.jpg}


\addsection{Strech Goals Neutral}{\covers/stretch_goals_neutral.jpg}

\begin{multicols}{3}

\small

\begin{itemize}[leftmargin=0pt, label={}, noitemsep, noitemsep]
  \item \textbf{\underline{Miniatures}} (47)
  \item
  \item \underline{Neutral Unit} (15)
  \item Boars (1)
  \item Rogues (1)
  \item Halflings (1)
  \item Peasants (1)
  \item Sharpshooters (1)
  \item Mummies (1)
  \item Nomads (1)
  \item Enchanters (1)
  \item Gold Golems (1)
  \item Diamond Golems (1)
  \item Trolls (1)
  \item Crystal Dragons (1)
  \item Azure Dragons (1)
  \item Rust Dragons (1)
  \item Faerie Dragons (1)
  \item
  \item \underline{Towns} (7)
  \item Castle (1)
  \item Dragon Utopia (1)
  \item Fortress (1)
  \item Dungeon (1)
  \item Tower (1)
  \item Ramparts (1)
  \item Necropolis (1)
  \item
  \item \underline{Others} (20)
  \item Campfire (4)
  \item Grail (1)
  \item Round Tracker (1)
  \item 5 Valuables (5)
  \item Treasure Chests (4)
  \item Morale (5)
  \item
  \item \underline{Plastic Tokens} (81)
  \item 1 Valuable (10)
  \item 3 Valuables (6)
  \item 1 Building Material (9)
  \item 3 Building Material (12)
  \item 10 Building Material (5)
  \item 1 Gold (11)
  \item 3 Gold (13)
  \item 10 Gold (10)
  \item 20 Gold (5)
\end{itemize}

\end{multicols}


\addsection{Battlefield}{\covers/battlefield.jpg}

\begin{multicols}{3}

\footnotesize

\begin{itemize}[leftmargin=0pt, label={}, noitemsep]
  \item \textbf{\small{\underline{Books Leaflets}}} (3)
  \item
  \item Battlefield Rulebook (1)
  \item Player's Aid Battlefield (2)
\end{itemize}

\begin{itemize}[leftmargin=0pt, label={}, noitemsep]
  \item \textbf{\small{\underline{Boards}}} (1)
  \item
  \item Battlefield (1)
\end{itemize}

\begin{itemize}[leftmargin=0pt, label={}, noitemsep]
  \item \textbf{\small{\underline{Obstacle Tokens}}} (10)
  \item
  \item Fire (2)
  \item Lake/Lake (1)
  \item Log/Castle Wall (2)
  \item Rock/Castle Wall (1)
  \item Skeleton, Stump/Castle Gate (1)
  \item Skeleton/Skeleton (1)
  \item Skull, Rock/Skull, Rock (1)
  \item Skull/Castle Wall (1)
\end{itemize}

\textbf{\small{\underline{Initiative Tokens}}} (1)

\vspace*{\fill}
\columnbreak

\begin{itemize}[leftmargin=0pt, label={}, noitemsep]
  \item \textbf{\small{\underline{Cards}}} (70)
  \item
  \item \textbf{Adventure Cards} (50)
  \item Campfire (2)
  \item Corpse (2)
  \item Crypt (2)
  \item Cyclops Stockpile (2)
  \item Dragon Utopia (2)
  \item Dwarven Treasury (2)
  \item Imp Cache (2)
  \item Learning Stone (2)
  \item Magic Spring (2)
  \item Obelisk (4)
  \item Pandora's Box (2)
  \item Scholar (2)
  \item Shrine of Magic Gesture (2)
  \item Spell Scroll (2)
  \item Temple (2)
  \item Trading Post (6)
  \item Treasure Chest (2)
  \item Tree of Knowledge (2)
  \item Warrior's Tomb (2)
  \item Water Wheel (2)
  \item Windmill (2)
  \item Witch Hut (2)
\columnbreak
  \item \textbf{Negative Morale Cards} (10)
  \item Attack set one to -1 (1)
  \item Discard 1 card at random (2)
  \item Paralyze one unit (1)
  \item Reroll +1 on attack die (2)
  \item Roll one less die when doing 2 search or 2 treasure (1)
  \item Search(x) is search(1) (1)
  \item Skip unit activation on -1 attack die (1)
  \item Suffer -1 to next attack, defense or power roll (1)
  \item
  \item \textbf{Positive Morale Cards} (10)
  \item Remove paralyze one unit (1)
  \item +1 attack, +1 defense or +1 power on next combat (1)
  \item Discard an adventurer card and draw another (1)
  \item Discard any number of cards and draw as many (1)
  \item Do another search(x) (1)
  \item Draw a card on combat start (2)
  \item Next attack set a die to +1 (1)
  \item Reroll a die (2)
  \item
  \item \textbf{Other Cards} (1)
  \item Battlefield deck cards (1)
\end{itemize}

\end{multicols}

\vfill
\begin{center}
  \includegraphics[width=0.9\linewidth]{\images/battlefield.png}
\end{center}


\end{document}
