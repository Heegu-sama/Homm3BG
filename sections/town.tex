% !TeX spellcheck = en_US
\addsection{Town}{\skills/artillery.png}

\iftoggle{printable}{\vspace{-\baselineskip}}{}

\begin{multicols*}{2}

\pagetarget{Town}{Each}\index{Town} Faction has their own Town, located in the center of their Starting Tile.
The Town is your most important location, as many Scenarios \pagelink{End}{may end} if it's \pagelink{Categories}{Flagged} by an enemy Hero.\par
The contents of your Town and overall Faction status are represented by the Town Board.
It shows your currently built Buildings, Resource costs for future Buildings, your Resource incomes, and status of Town Action Tokens.\par
All Factions are able to Build the following Buildings\index{Buildings} in their Town:
\begin{itemize}
  \item \textbf{City Hall} – Provides Resource income or a Faction-Specific Ability.
  \item \textbf{Citadel} – Allows you to Reinforce Units when using the Population Token.
Also \pagelink{Walls}{protects your Town} when it is attacked.
  \item \textbf{Unit Dwellings} – Allows you to Recruit Units.
Dwellings have three Levels that unlock new Units, which must be Built in the following order: \svgunit{bronze}\svgunit{silver}\svgunit{golden}
  \item \textbf{Mage Guild}\index{Mage Guild} - gains you \pagelink{spells}{Spells}.
  \item \textbf{Faction Building} - a Faction-Specific Building with a unique effect.
\end{itemize}
\textbf{One Building} may be Built each Round by using the Build Token\index{Build Token}.
When you build a Building, pay its cost in Resources, flip the Build Token to its inactive side, and place the new Building’s Cardboard Piece into its proper slot on the Town Board.
If the Building has any immediate effects upon Building it, resolve them now.\par
Built Buildings are always represented by a symbol within a circle.
Buildings that can be built in the future are represented by a rectangle that contains the Building's cost in Resources.
Some Building Tiles are double-sided, and may later be upgraded and flipped to represent two different buildings at the same time. Such upgrades must be \textbf{Built in order}.\par

\vspace*{\fill}

\begin{center}
  \transparent{0.3}\includegraphics[width=\linewidth]{\art/earthquake.png}
\end{center}

\vspace*{\fill}

\end{multicols*}
