% !TeX spellcheck = en_US
\addsection{Map Elements}{\skills/logistics.png}

\iftoggle{printable}{\vspace{-\baselineskip}}{}

\begin{multicols*}{2}
Each Scenario is built using four types of \pagetarget{Map}{Map} Tiles.
Players start on their Faction-Specific Starting (I) Tile.
Other tiles may be placed and discovered as described on the next page.
During the game's setup, all face-down tiles should be selected randomly from the pool of possible Tiles as described by the Scenario and shuffled, keeping them face-down.\par
The \textbf{roman numeral} on each tile describes the overall \textbf{difficulty of Neutral Units} on that tile, as well as the number of rewards players can expect to find on that Tile.
Starting (I) Tiles are the easiest while Center (VI–VII) Tiles are the most difficult.\par

\vfill
\begin{center}
  \transparent{0.2}\includegraphics[width=0.6\linewidth]{\art/counterstrike.png}
\end{center}
\vfill

\subsection*{Map Tile Anatomy\index{Map Tile}}
Each Map Tile is divided into 7 separate \textbf{Fields} that your Heroes can \textbf{Visit}.
When your Hero moves to a Field, they must immediately Visit it, or
first start a \pagelink{Combat}{Combat} against the enemies guarding it before Visiting.
Empty Fields\index{Empty Field} do nothing when Visited.
Solid yellow lines on a Field's edge cannot be passed through.
\pagelink{Difficulty}{Roman numerals} written on a Field indicate that the Field is guarded by Neutral enemies that must be fought to Visit it.\par
\columnbreak
\includegraphics[width=\linewidth]{\images/maptiles.png}
\begin{itemize}
  \footnotesize
  \item[\textbf{1.}] Starting Map Tiles: I
  \item[\textbf{2.}] Far Map Tiles: II-III
  \item[\textbf{3.}] Near Map Tiles: IV-V
  \item[\textbf{4.}] Center Map Tiles: VI-VII
\end{itemize}

\vfill
\begin{center}
  \begin{scriptsize}
  \begin{tikzpicture}
    \draw (0, 0) node[inner sep=0] {\makebox[\linewidth][c]{\includegraphics[width=0.75\linewidth]{\images/fields.png}}};
    \draw (0, -0.3) node {\encircle{\phantom{.}1\phantom{.}}};
    \draw (-1.9, 0) node {\encircle{\phantom{.}2\phantom{.}}};
    \draw (2.3, 0) node {\encircle{\phantom{.}2\phantom{.}}};
    \draw (-1.4, 1.7) node {\encircle{\phantom{.}2\phantom{.}}};
    \draw (0.6, -1.9) node {\encircle{\phantom{.}2\phantom{.}}};
    \draw (0.5, 1.7) node {\encircle{\phantom{.}2\phantom{.}}};
    \draw (1.4, -2.2) node {\encircle{\phantom{.}3\phantom{.}}};
    \draw (1.4, 2.5) node {\encircle{\phantom{.}4\phantom{.}}};
    \draw (1.4, -1) node {\encircle{\phantom{.}4\phantom{.}}};
    \draw (-1, 0.5) node {\encircle{\phantom{.}5\phantom{.}}};
    \draw (-1.1, -1.1) node {\encircle{\phantom{.}6\phantom{.}}};
    \draw (-1.4, -1.7) node {\encircle{\phantom{.}7\phantom{.}}};
  \end{tikzpicture}
  \end{scriptsize}
\end{center}

\begin{itemize}
  \footnotesize
  \begin{multicols}{2}
    \item[\textbf{1.}] Empty Field
    \item[\textbf{2.}] Location
    \item[\textbf{3.}] Artifact Symbol
    \item[\textbf{4.}] Field Difficulty
    \item[\textbf{5.}] Border line
    \item[\textbf{6.}] Blocked Field
    \item[\textbf{7.}] Tile name like \\{} \texttt{F2} or \texttt{\#N3}
  \end{multicols}
\end{itemize}

\note{3}{
  Blocked Field cannot be entered, but can be exited.
}
\end{multicols*}

\clearpage

\begin{multicols}{2}

\subsection*{\pagetarget{Categories}{Location Categories}}
Visiting Fields provides Heroes with benefits, such as gaining Resources or Cards (see \pagelink{All Map Locations}{All Map Locations}).
There are three categories of Fields:
\begin{itemize}
  \item \textbf{Visitable} – Once you Visit this field, place a Black Cube on it.
    Treat it as an Empty Field as long as it has a Black Cube.
  \item \textbf{Flaggable} – These Fields can be captured by players and provide passive benefits.
    When you Visit one, place your Faction Cube on it.
    Enemy Heroes who Visit your Flagged Fields will replace your Cube with theirs to \textbf{steal} the Field’s effects.
    Allied Heroes treat Flagged Fields \textbf{as if they were empty}.
  \item \textbf{Revisitable} – You can Visit this Field multiple times.
    Do not place any Cubes when you Visit it.
    You may pay 1 MP to Visit this Field again.
\end{itemize}

\subsection*{\pagetarget{Placing}{Placing and Discovering New Tiles}}\index{Discovering Tiles}
Heroes may spend 1 MP to either reveal an adjacent face-down Tile, or to place a Far (II–III) Map Tile from their own supply of Tiles (scenario rules determine the Tiles in your own supply).
All face-down Tiles should be kept \textbf{hidden from all players} until they are about to be placed or revealed.
New tiles must be placed adjacent to the Hero who spends the MP, and connected to at least two other existing Tiles.
New Tiles must also be positioned so that there is a valid path that eventually connects them with all other Tiles.
You may always rotate Map Tiles when placing or revealing them.

\medskip
\note{6}{
  When you Visit a Visitable field, \textbf{you must} place a black cube on that Field even if you cannot or choose not to use that Field's effects.
}

\end{multicols}

\vspace*{-1em}
\begin{figure*}[!hb]
  \centering
  \includegraphics[width=\textwidth]{\images/placement.png}
  \begin{tikzpicture}[overlay]
    \node at (5, 10) {\footnotesize{\textbf{\textit{\textcolor{darkcandyapplered}{You cannot add this tile here}}}}};
    \node at (5, 1.4) {\footnotesize{\textbf{\textit{\textcolor{cadmiumgreen}{You can add this tile here, because it will}}}}};
    \node at (3.9, 1) {\footnotesize{\textbf{\textit{\textcolor{cadmiumgreen}{be adjacent to two other tiles}}}}};
  \end{tikzpicture}
\end{figure*}

\clearpage

\subsection*{Example Turn}

\begin{multicols*}{2}

\textit{Alice wants to capture an adjacent \pagelink{Mines}{Mine} by Flagging it with her Main Hero, Sandro the Necromancer.
    She spends 1 MP to move onto the Mine, which begins \pagelink{Combat}{Combat} against Neutral Units, since the Field has a \pagelink{Difficulty}{Difficulty Rating} and has not been previously Flagged by any player.}\par

\includegraphics[width=1.1\linewidth]{\examples/sandro_takes_mine.png}

\textit{The Mine turns out to be guarded by Troglodytes, which have 3 HP \svg{health_points}.
Alice's current hand consists of a Power Card, a Lightning Bolt, Haste, and a Town Portal.
During the Combat, she casts the Lightning Bolt, and Empowers \svg{empower} it with Haste's alternative (bottom) effect, which makes the Lightning Bolt deal 3 damage \svg{damage}, killing the Troglodytes and winning the Combat.}

\includegraphics[width=\linewidth]{\examples/sandro_empowering_lightning_bolt.png}

\columnbreak
\textit{The Combat lasted for only one Round, so Alice would not have been able to cast both Lightning Bolt and Haste, since players are limited to playing only one Spell Card per Combat Round.}\par


\textit{Alice now Flags the Mine by placing one of her Faction Cubes on it.
    Flagging this particular Mine increases her Building Materials \svg{building_materials} production by 2, and she also immediately gains the Mine's production value of 2 \svg{building_materials} as she was the first player to Flag it.}\par
\textit{Afterwards, Alice wants to go back to defend a previously Flagged Settlement by casting the Town Portal still left in her hand.
    Her Hero is Level 2, so she can empower it with the Power Card's Expert Effect \svgeven{expert}, which grants her an additional Movement Point after casting it.
}

\includegraphics[width=\linewidth]{\examples/sandro_empowering_town_portal.png}

\vfill
\hspace{2em}
{\transparent{0.2}\includegraphics[width=\linewidth]{\art/resurrection.png}}

\begin{expansion}{cove}

\subsection*{Sea Map Tiles}
Sea Map Tiles have their unique backside and during setup they will placed as shown in the chosen scenario map layout. Discovering sea map tiles work normally.\par
Sea Map Tiles may contain both land and sea fields. Movement from land to land, from sea to sea or from sea to land work as normal. But when your hero \textbf{enters a sea field from a land field}, their movement for the turn ends. You may use unspent MP for other actions (for example to extend neutral Combats), but you can't move anymore that turn, not even with the help of card effects.

\bigskip
\begin{multicols*}{2}
\hspace{-0.5em}\includegraphics[width=1.15\linewidth]{\images/map-tile-cove.png}
\columnbreak
\hspace{-0.5em}\includegraphics[width=1.15\linewidth]{\images/map-tile-sea.png}
\end{multicols*}
\medskip
\begin{center}
  \footnotesize{\textbf{\textit{\textcolor{darkcandyapplered}{Sea Map Tile}}}}
\end{center}
\medskip

\subsection*{Whirlpool Tokens}

Whirlpool Location tokens allow Heroes to travel between them. If your Hero enters a Whirlpool, move him immediately to another Whirlpool Token. If there are more valid targets, roll an Attack die to determine where it takes your hero. Reroll any die, which shows the number, your hero is standing on. After each such travel, lose 1 unit from your Unit Deck. You can recruit it again later. %What is losing mean: complete unit card? Or reduce pack to few?
\end{expansion}

\begin{expansion}{stronghold}

\subsection*{Subterranean Map Tiles}

Subterranean Map Tiles have their unique backside and will be place during setup as the chosen scenario map layout is shown. They work similar with regular map tiles, except that you cannot move between a Surface and a Subterranean tile without using a Subterranean Gate or a Town Portal Spell. No other move effects from cards can allow you to move from one to the other. % NOTE: Discovering from one to the other should also impossible.

\bigskip
\begin{center}
  \begin{scriptsize}
    \centering
    \begin{tikzpicture}
      \node at (0, 0) {\includegraphics[width=0.5\textwidth]{\images/map-tile-stronghold.png}};
      \node at (0, 2.2) {\footnotesize{\textbf{\textit{\textcolor{darkcandyapplered}{Stronghold Map Tile}}}}};
      \node at (3.21, -1.11) {\includegraphics[width=0.5\textwidth]{\images/map-tile-sub.png}};
      \node at (3.21, -3.41) {\footnotesize{\textbf{\textit{\textcolor{darkcandyapplered}{Subterranean Map Tile}}}}};
      \node[rotate=-60] (gate) at (1.6, -0.53) {\includegraphics[width=0.32\textwidth]{\images/subter-gate-outline.png}};
      \node at (3.21, 1.11) {\footnotesize{\textbf{\textit{\textcolor{darkcandyapplered}{Subterranean Gate}}}}};
    \end{tikzpicture}
  \end{scriptsize}
\end{center}

\subsection*{Subterranean Gate}
The Subterranean Gate is a Location Token, which represents the way in and out of a Subterranean tile. After discovering a tile with a Subterranean Gate on it, \textbf{choose a field} that is adjacent to the Subterranean tile and place the Subterranean Gate token on both this field and an adjacent field of the Subterranean tile (each half of the token is placed on a different tile). The scenario map layout only shows \textbf{which two tiles are connected} by the Subterranean Gate, \textbf{not the exact fields that must be connected}.\par
When a Hero enters a Subterranean Gate, discover the Map tile on the other side for free and place it normally, choosing the field that will be replaced by the Subterranean Gate token.
Both fields covered by the Subterranean Gate token are treated as one field. Once both ends of the Subterranean Gate are placed, it allows traveling both ways.

\end{expansion}

\subsection*{Location Tokens}
Location Tokens are added by several expansion to the game and they will be placed on specific fields of the map, shown in the scenario's map layout. \textbf{Unless otherwise stated}, the following rules apply to all location tokens:
\begin{itemize}
  \item When you discover a Tile with a Location Token on it, place the Tile normally and then place the Token on the field indicated by the scenario's map layout.
  \item Placed tokens overwrite the original location on their field.
  \item Token cannot be placed on blocked fields, other tokens, or fields containing locations required to meet any of the scenario's victory conditions. Keep these rules in mind when discovering and rotate a tile with a token on it.
\end{itemize}
\vspace*{\fill}

\begin{center}
  \includegraphics[width=\linewidth]{\art/dendroid.jpg}
\end{center}

\end{multicols*}
