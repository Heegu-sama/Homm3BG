% !TeX spellcheck = en_US
\en{
  \addsection{Introduction}{\spells/magic_arrow.png}
}
\pl{
  \addsection{Wprowadzenie}{\spells/magic_arrow.png}
}

\bigbreak

\en{
  \hypertarget{Heroes of Might and Magic III}{\textbf{Heroes of Might and Magic III: The Board Game}} is a tactical strategy RPG board game for 1-3 players using the core box set.
  The continent of Antagarich is at war as several different Factions, led by their Heroes, battle for supremacy.
  Choose your Faction and Hero and banish your unruly enemies from these lands!
}
\pl{
  \hypertarget{Heroes of Might and Magic III}{\textbf{Heroes of Might and Magic III: The Board Game}} to taktyczna strategiczna gra planszowa RPG dla 1-3 graczy korzystających z zestawu podstawowego.
  Kontynent Antagarich jest w stanie wojny, a kilka różnych frakcji, prowadzonych przez swoich bohaterów, walczy o dominację.
  Wybierz swoją frakcję i bohatera i wypędź zuchwałych wrogów z tych ziem!
}

\begin{multicols}{2}
\en{
  \textbf{Notice}: In this rule book, game and component terms are Capitalized.
  \textbf{Bold text} is used to draw attention to important rules.
  \textit{Italicization} is used for gameplay examples.
  \hyperlink{Heroes of Might and Magic III}{Brown colored hyperlinks} will take you to other parts of the rule book.
}
\pl{
  \textbf{Uwaga}: W niniejszej instrukcji terminy związane z grą i komponentami są pisane Wielką literą.
  \textbf{Tekst pogrubiony} jest używany w celu zwrócenia uwagi na ważne zasady.
  \textit{Kursywa} jest używana w przykładach rozgrywki.
  \hyperlink{Heroes of Might and Magic III} {Brązowe hiperłącza} prowadzą do innych części instrukcji.}
% The following glyphs MUST be rendered before their first use in tabular environment in all_map_locations.tex.
% Inkscape will fail otherwise.
% Other glyphs referenced over there ONLY cause no issues ¯\_(ツ)_/¯
\phantom{
  \includesvg[height=0.1px]{\svgs/artifact.svg}
  \includesvg[height=0.1px]{\svgs/movement.svg}
  \includesvg[height=0.1px]{\svgs/spellpower.svg}
  \includesvg[height=0.1px]{\svgs/treasure.svg}
}
\pl{ % Workaround for Inkscape bug. To be deleted once translation is complete.
  \phantom{
    \includesvg[height=0.1px]{\svgs/treasure.svg}
    \includesvg[height=0.1px]{\svgs/morale_positive.svg}
    \includesvg[height=0.1px]{\svgs/treasure.svg}
    \includesvg[height=0.1px]{\svgs/resource_die.svg}
    \includesvg[height=0.1px]{\svgs/experience.svg}
    \includesvg[height=0.1px]{\svgs/spellpower.svg}
    \includesvg[height=0.1px]{\svgs/artifact.svg}
    \includesvg[height=0.1px]{\svgs/morale_positive.svg}
    \includesvg[height=0.1px]{\svgs/morale_negative.svg}
    \includesvg[height=0.1px]{\svgs/movement.svg}
  }
}
\vfill
\columnbreak
\en{
  \note{10}{Exceptions and notes with \hyperlink{Heroes of Might and Magic III}{amber colored hyperlinks} are explained in boxes like this one.
    \medskip\\
    Conflicting rule changes on components follow this priority: Player Cards, Unit Cards, Town Boards, Mission Book, this rule book.
  }
}
\pl{
  \note{10} {Wyjątki i uwagi z \hyperlink{Heroes of Might and Magic III} {bursztynowymi hiperłączami} są wyjaśnione w ramkach takich jak ta.
    \medskip\\
    Konflikty zasad mają następujący priorytet: Karty Graczy, Karty Jednostek, Plansze Miast, Księga Misji, niniejsza instrukcja.
  }
}
\vfill
\end{multicols}

\begin{scaledfigure}[blanker]
  \centering
  \includegraphics[width=\linewidth, height=\myspace, keepaspectratio]{\art/hydra.png}
\end{scaledfigure}
