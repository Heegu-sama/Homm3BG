% !TeX spellcheck = en_US

\pagetarget{Card Effect Keywords}{\addsection{Card Effect Keywords}{\skills/interference.png}}

Most expansions add new cards to the game. Some of these cards use keywords with more elaborate rules and require specific game components.
This section provides the rules for these keywords.

\pagetarget{spell keywords}{\subheader{Spell Card keywords}}

Some Spell cards allow you to place \textbf{Effects} or \textbf{Summoned Units} on the Combat Board.
Use the corresponding Tokens to mark these spaces.
\medskip

\begin{expansion}{cove}
  \begin{multicols}{2}
    \subsection*{Clone Token}
    \setlength\intextsep{0pt}
    \setlength\columnsep{1em}
    \begin{wrapfigure}{r}{0.3\linewidth}
      \includegraphics[width=0.6\linewidth]{\images/clone-token.png}\vspace*{1em}
      \includegraphics[width=\linewidth]{\images/clone-token.png}
      \centering\footnotesize\imagecaption{Clone \mbox{Tokens}}
    \end{wrapfigure}
    When you cast the \textbf{Clone Spell}, choose an allied unit (depending on the \svg{empower} used) and an adjacent empty space on the Combat Board.
    Place the small Clone Token on the chosen unit and the larger Token of the same color (blue or red) on the empty space.
    The larger Token on the empty space represents the Clone.
    The Clone has the same statistics and special abilities as the original unit, except its \svg{health_points} is 1 and its \svg{defense} is 0.
    These values cannot be increased by any effects.  % what about other statistics?
    The Clone perishes if it is attacked (even without taking damage) or takes any \svg{damage} from a Spell.  % clarification?
    If the original unit is removed from the Combat Board, remove the Clone as well.
    The Clone does not inherit effects played on the original unit.
    \vfill
    \columnbreak

    \subsection*{\textit{Clone Example}}
    \textit{%
      One of your units in Combat is ``Few'' Griffins with an Initiative value of 6.
      You cast the Clone Spell on them, placing the small Clone Token on the Griffins and the larger Token of the same color on an adjacent empty space.
      Later, you cast ``Haste'' with 0 \svg{empower} on the original Griffins, increasing their Initiative by 1.
      The original Griffins' Initiative value is now 7, but the Clone's Initiative remains 6.
    }
    \vspace*{1em}
    \par\noindent
    \begin{tikzpicture}
      \draw (0, 0) node[rotate=180] {\includegraphics[width=\linewidth]{\images/combat_board_half.png}};
      \draw (-1, 1.3) node {\includegraphics[width=0.22\linewidth]{\cards/unit-few.png}};
      \draw (-0.8, 1.2) node[circular drop shadow={shadow scale=0.9, shadow xshift=0.3ex, shadow yshift=-0.3ex, opacity=0.7,
        fill=black, path fading={circle with fuzzy edge 15 percent},
        every shadow}] {\includegraphics[width=0.1\linewidth]{\images/clone-token-2.png}};
      \draw (1, 1.3) node[circular drop shadow={shadow scale=0.9, shadow xshift=0.3ex, shadow yshift=-0.3ex, opacity=0.7,
        fill=black, path fading={circle with fuzzy edge 15 percent},
        every shadow}] {\includegraphics[width=0.18\linewidth]{\images/clone-token-2.png}};
      \draw (-0.4, -1.7) node {\includegraphics[width=0.22\linewidth]{\cards/haste.png}};
    \end{tikzpicture}
  \end{multicols}
  \vspace*{-6em}
  {\hspace{1em}\transparent{0.2}\includegraphics[width=0.5\linewidth]{\art/magic_arrow.png}\hfill}
\end{expansion}

\newpage
\begin{multicols}{2}
\begin{expansion}{stronghold}
  \subsection*{Quicksand Token}
  Depending on the \svg{empower} used, take 2, 4 or 6 \textbf{Quicksand Tokens} from the same set (blue or red).
  Half of them must be ``empty'' and half must have the Quicksand symbol.
  Shuffle the Tokens face down and place one on each chosen empty space on the Combat Board.
  If there are not enough empty spaces to place all of them, discard any leftover Tokens.
  Once placed, you may look at your Tokens at any time.
  When a unit enters a space with a Quicksand Token, reveal that Token.
  If it is empty, the unit continues its movement.
  If the Token shows the Quicksand icon, the unit's movement and activation end immediately.

  \textbf{Note:} two Quicksand tokens can occupy a single space only if the tokens are placed by different players.
  \bigskip
  \begin{multicols*}{2}
    \centering
    \footnotesize
    \includegraphics[width=0.5\linewidth]{\images/quicksand-token.png}\\
    \imagecaption{Quicksand Token}

    \columnbreak
    \includegraphics[width=0.5\linewidth]{\images/quicksand-token-2.png}\\
    \imagecaption{Empty Quicksand Token}
  \end{multicols*}
\end{expansion}

\vspace*{1em}
\begin{expansion}{battlefield,conflux}
  \subsection*{Firewall Token}
  The \textbf{Fire Wall Spell} and \textbf{Luna's Specialty cards I and VI} instruct you to place the Fire Wall card onto a space on the Combat Board.
  You can use the Firewall Tokens instead of the cards.
  Place the Spell/Specialty card next to the Combat Board as a reminder.
  When the card is removed or discarded, discard the token, too.
  When using the large \pagelink{Battlefield Combat}{Battlefield Board}, you must place the Token rather than the card.

  \medskip
  \begin{multicols}{2}
    \centering
    \footnotesize
    \includegraphics[width=0.7\linewidth]{\images/firewall-bf-token.png}
    \imagecaption{Firewall Token\\(Spell)}

    \columnbreak
    \includegraphics[width=0.7\linewidth]{\images/firewall-vi-token.png}
    \imagecaption{Firewall Token\\(Luna's Specialty)}
  \end{multicols}
\end{expansion}

\columnbreak
\begin{expansion}{stretchgoals2}
  \subsection*{Land Mine Token}
  Depending on the \svg{empower} used, take 2, 4 or 6 \textbf{Land Mine tokens} from the same set  (blue or red).
  Half of them must be ``empty'' and half of them have a +2 \svg{damage} symbol.
  Shuffle the tokens face down and place one each on chosen empty spaces on the Combat Board.
  If there are not enough empty spaces left to place all of them, discard any leftover Tokens.
  Once they are placed, you can always look at your Tokens.
  When a unit enters a space with a Land Mine Token, reveal that Token.
  If it is empty, the unit continues its movement, but if the Token shows the +2 \svg{damage} symbol, the unit takes 2 damage.
  If it is still one the Board, it continues its activation.

  \textbf{Note:} two Land Mine tokens can occupy a single space only if the tokens are placed by different players.
  \bigskip
  \begin{multicols}{3}
    \centering
    \footnotesize
    \includegraphics[width=0.9\linewidth]{\images/land-mine-token.png}
    \imagecaption{Landmine Token\\(Backside)}

    \columnbreak
    \includegraphics[width=0.9\linewidth]{\images/land-mine-token-0.png}
    \imagecaption{Empty Landmine Token}

    \columnbreak
    \includegraphics[width=0.9\linewidth]{\images/land-mine-token-2.png}
    \imagecaption{Landmine Token}
  \end{multicols}
\end{expansion}

\vspace*{1em}
\begin{expansion}{stretchgoals2}
  \subsection*{Force Field Token}
  \setlength\intextsep{0pt}
  \setlength\columnsep{1em}
  \begin{wrapfigure}{r}{0.2\linewidth}
    \includegraphics[width=\linewidth]{\images/force-field-token.png}
    \begin{center}
      \footnotesize
      \imagecaption{Force Field Token}\\
      \vspace*{1em}
    \end{center}
  \end{wrapfigure}
  Use the \textbf{Force Field Token} to mark the effect of the \textbf{Force Field Spell}.
  The token represents an Obstacle, so only \svg{unit_flying} units can move through a space with this Token.
  If you're using the normal Combat Board, place only 1 Force Field Token, but if you're using the \pagelink{Battlefield Combat}{Battlefield Board}, place 2 Tokens on two adjacent spaces (instead of one).
\end{expansion}
\end{multicols}

\clearpage
\pagetarget{Summoning}{\subheader{Summoning}\index{Summoning}}
\begin{expansion}{inferno,conflux}
  Summoned units activate in the round they were summoned if their initiative is lower or equal to the initiative of the currently activated unit.
  Otherwise, treat them as if they already activated this Combat round.
\end{expansion}
\bigskip
\begin{multicols}{2}
  \begin{expansion}{conflux}
    \subsection*{Summon Elemental Units}
    The \textbf{Summon Elemental Spells} allow you to add a Elemental unit to the Combat Board during Combat.
    Keep the Summon unit card Deck face up near the neutral Unit Decks.
    Depending on the \svg{empower} used, you can Summon either a non-reinforced unit (a Few) or a reinforced one (a Pack).
    Take the corresponding unit card from the Deck of the Summoned units and place it on a chosen empty space.
    You can decide to place the \textbf{Summon Unit Tokens} instead of the cards.
    Using the battlefield Board you \textbf{must} place the Token.
    Summoned units remain under your control until they perish or the Combat ends, whichever comes first.
    A Summoned Elemental Unit is \textbf{not} added to your Unit Deck.
    \bigskip

    \begin{multicols}{2}
      \centering
      \footnotesize
      \includegraphics[width=0.9\linewidth]{\cards/unit-summoned-few.png}\\
      \imagecaption{A Few Fire Elementals}

      \smallskip
      \includegraphics[width=0.5\linewidth]{\images/summon-token-fire.png}\\
      \imagecaption{Fire Elemental Token}

      \smallskip
      \includegraphics[width=0.5\linewidth]{\images/summon-token-earth.png}\\
      \imagecaption{Earth Elemental Token}

      \columnbreak
      \includegraphics[width=0.9\linewidth]{\cards/unit-summoned-pack.png}\\
      \imagecaption{A Pack of Fire Elementals}

      \smallskip
      \includegraphics[width=0.5\linewidth]{\images/summon-token-water.png}\\
      \imagecaption{Water Elemental Token}

      \smallskip
      \includegraphics[width=0.5\linewidth]{\images/summon-token-air.png}\\
      \imagecaption{Air Elemental Token}
    \end{multicols}
  \end{expansion}
  \columnbreak

  \begin{expansion}{inferno}
    \subsection*{Summon Demons}
    The \textbf{Pit Lords unit} from the Inferno Faction may use its \svg{unit_other} ability to summon or reinforce the Demon units during Combat.
    This effect cannot summon Demons from the Neutral Unit Deck.
    Place the summoned Demon unit adjacent to the Pit Lords. If a Few Demons are already in Combat, reinforce these Demons instead of summon them.
    After Combat, unless stated otherwise, the summoned Demons \textbf{are} added to your Unit Deck.
    \bigskip

    \begin{multicols}{2}
      \centering
      \footnotesize
      \includegraphics[width=0.9\linewidth]{\cards/unit-pitlord.png}\\
      \imagecaption{Pit Lords}

      \columnbreak
      \includegraphics[width=0.9\linewidth]{\cards/unit-demon.png}\\
      \imagecaption{Demons}
    \end{multicols}
  \vspace*{2.4em}
  {\hfill\transparent{0.2}\includegraphics[width=0.9\linewidth]{\art/cerberi.png}}
  \vspace*{2.4em}
  \end{expansion}

\end{multicols}


\clearpage
\pagetarget{unit keywords}{\subheader{Unit Ability keywords}}

The following components or keywords belong to new unit abilities. The Tokens are usually placed on other units and influence their statistics.
\begin{multicols}{2}
  \begin{expansion}{cove,stretchgoals2}
    \subsection*{Weakness Token}
    \setlength\intextsep{0pt}
    \setlength\columnsep{1em}
    \begin{wrapfigure}{r}{0.25\linewidth}
      \includegraphics[width=0.8\linewidth]{\images/weakness-token.png}
      \centering
      \footnotesize
      \imagecaption{\mbox{Weakness} Token}
      \vspace{1em}
    \end{wrapfigure}
    The \textbf{Few Sorceresses unit} from the Cove Faction can use its \svg{unit_other} abilty to place a Weakness Token on any unit.
    A \textbf{Pack Sorceresses} deal that Token automatically to their defending unit because of their \svg{unit_attack} ability.
    A unit with this Token suffers $-1$~\svg{attack} or $-2$~\svg{attack}, as indicated by the Token's side.
    Each unit can have only one such Token at a time.
    If a unit would gain a second one, the player controlling it chooses which one to keep.
    \begin{center}
      \vspace*{2.5em}
      {\transparent{0.2}\includegraphics[width=0.9\linewidth]{\art/weakness.png}}
    \end{center}
  \end{expansion}
  \columnbreak

  \begin{expansion}{stronghold,stretchgoals2}
    \subsection*{Attack Token}
    \setlength\intextsep{0pt}
    \setlength\columnsep{1em}
    \begin{wrapfigure}{r}{0.2\linewidth}
      \includegraphics[width=\linewidth]{\images/attack-token.png}
      \centering
      \footnotesize
      \imagecaption{Attack Token}
      \vspace{1em}
    \end{wrapfigure}
    The \textbf{Orge unit} from the Stronghold Faction can use its \svg{unit_other} ability to place Attack Tokens on other allied unit cards.
    Units with such a token gains an additional +1 \svg{attack} or +2 \svg{attack}, as indicated by the Tokens side.
    Each unit can have only one such Token at a time.
    If a unit would gain a second one, the player controlling it chooses which one to keep.
  \end{expansion}

  \vspace*{1em}
  \begin{expansion}{stronghold}
    \subsection*{Corrosion Token}
    \setlength\intextsep{0pt}
    \setlength\columnsep{1em}
    \begin{wrapfigure}{r}{0.25\linewidth}
      \includegraphics[width=0.8\linewidth]{\images/corrosion-token.png}
      \centering
      \footnotesize
      \imagecaption{\mbox{Corrosion} Token}
      \vspace{1em}
    \end{wrapfigure}
    The \textbf{Behemoth unit} from the Stronghold Faction can use his \svg{unit_attack} ability to place a Corrosion Token on the defending units, which reduces their \svg{defense} by 1 to a minimum of ``0''.
    A unit can have only one such Token at a time.
    If a unit should gain a second Token, discard one of the Tokens. Unless removed by Spells or other effects, a Corrosion Token stays on the unit until end of combat.
  \end{expansion}
\end{multicols}

\vfill
\subheader{Fan-Made Database}
\begin{wrapfigure}{r}{0.2\textwidth}
  \vspace{-1.5em}
  \includegraphics[width=\linewidth]{\qr/wiki-database.png}
\end{wrapfigure}

Many cards in \textbf{Heroes of Might and Magic 3: The Board Game} have ambiguous or misleading wordings.
Fortunately the developers of the game have clarified a lot of card effects.

The Community maintains a \textbf{Fanmade Database}: a collection of nearly all components of the game with tons of notes for a better understanding of specific rules.
If you have questions about specific cards, visit this website:
\mbox{\href{https://homm3bg.wiki/}{https://homm3bg.wiki/}}
\vfill
