% !TeX spellcheck = en_US

\addsection{Detailed Card Effects}{\skills/interference.png}

Most expansions add new Cards to the game, some of which follow more extensive rules and also require their own specific game components.
You find all these specific rules in this section, sorted by keywords or component designation. Additionally, the required components are visualized and the specific Card titles are \textbf{fat printed} in the description.

\begin{multicols}{2}

\subsection*{New Spell Tokens}
The new Spell Cards let you usually place \textbf{Effect Obstacles} or \textbf{Summoned Units} on the Combat Board. For each Obstacle or Unit exist a specific Token, you have to use to mark the affected spaces.

\begin{expansion}{tower,battlefield,conflux}
    \subsection*{Firewall Token}
    The \textbf{Fire Wall Spell} and \textbf{Luna's Specialty Card IV} instructed you to place the Fire Wall Card onto a space on the Combat Board.
    You can use the Firewall Tokens instead of the Cards.
    Place the Spell Card next to the Combat Board as a reminder.
    When the Card is removed or discarded, discard the token, too.
    Using the large Battlefield Board, you must place the Token.
    \begin{multicols}{2}
        \centering
        \includegraphics[width=0.9\linewidth]{\images/firewall-vi-token.png}
        \textbf{\scriptsize\color{darkcandyapplered}Firewall Token\\ (Spell Card)}

        \columnbreak
        \includegraphics[width=0.9\linewidth]{\images/firewall-vi-token.png}\\
        \textbf{\scriptsize\color{darkcandyapplered}Firewall Token\\ (Specialty Card)}
    \end{multicols}
\end{expansion}

\begin{center}
    \includegraphics[width=0.8\linewidth]{\art/tome_of_earth_magic.png}
\end{center}
\columnbreak
\begin{expansion}{stretchgoals2}
    \subsection*{Force Field Token}
    \setlength\intextsep{0pt}
    \setlength\columnsep{1em}
    \begin{wrapfigure}{r}{0.2\linewidth}
        \includegraphics[width=\linewidth]{\images/force-field-token.png}
    \end{wrapfigure}
    Use the \textbf{Force Field Token} to mark the effect of the \textbf{Force Field Spell}.
    The token represents an Obstacle, so only \svg{unit_flying} Units can move through a space with this Token.
    Place only 1 Force Field Token on the normal Combat Board, but if you use the Battlefield Board place 2 Tokens on two adjacent spaces (instead of one).
\end{expansion}

\vspace*{1em}
\begin{expansion}{stronghold}
    \subsection*{Quicksand Tokens}
    Depending on the \svg{empower} used, take 2, 4 or 6 \textbf{Quicksand tokens} from the same set (blue or red).
    Half of them must be "empty" and half of them must have the Quicksand symbol.
    Shuffle the Tokens face down and place one each on chosen empty spaces on the Combat Board.
    If there are not enough empty spaces left to place all of them, discard any leftover Tokens.
    Once they are placed, you can always look at your Tokens.
    When a Unit enters a space with a Quicksand Token, reveal that Token.
    If it is empty, the unit continues its Movement, but if the Token shows the Quicksand icon, the Unit's movement and activation ends.
    \bigskip
    \begin{multicols*}{2}
        \centering
        \includegraphics[width=0.6\linewidth]{\images/quicksand-token.png}\\
        \textbf{\scriptsize\color{darkcandyapplered}Quicksand Token\\}

        \columnbreak
        \includegraphics[width=0.6\linewidth]{\images/quicksand-token-2.png}\\
        \textbf{\scriptsize\color{darkcandyapplered}Empty Quicksand\\Token\\}
    \end{multicols*}
\end{expansion}

\vspace*{\fill}
\columnbreak

\begin{expansion}{stretchgoals2}
    \subsection*{Land Mine Tokens}
    Depending on the \svg{empower} used, take 2, 4 or 6 \textbf{Land Mine tokens} from the same set  (blue or red).
    Half of them must be "empty" and half of them have a +2 \svg{damage} symbol.
    Shuffle the tokens face down and place one each on chosen empty spaces on the Combat Board.
    If there are not enough empty spaces left to place all of them, discard any leftover Tokens.
    Once they are placed, you can always look at your Tokens.
    When a Unit enters a space with a Land Mine Token, reveal that Token.
    If it is empty, the Unit continues its movement, but if the Token shows the +2 \svg{damage} symbol, the unit takes 2 damage.
    If it is still one the Board, it continues its activation.
    \bigskip
    \begin{multicols}{3}
        \centering
        \includegraphics[width=0.9\linewidth]{\images/land-mine-token.png}
        \textbf{\scriptsize\color{darkcandyapplered}Landmine Token\\ (Backside)}

        \columnbreak
        \includegraphics[width=0.9\linewidth]{\images/land-mine-token-0.png}\\
        \textbf{\scriptsize\color{darkcandyapplered}Empty Landmine Token\\}

        \columnbreak
        \includegraphics[width=0.9\linewidth]{\images/land-mine-token-2.png}\\
        \textbf{\scriptsize\color{darkcandyapplered}Landmine Token\\}
    \end{multicols}
\end{expansion}
\columnbreak

\begin{expansion}{cove}
    \subsection*{Clone Spell Card}
    \setlength\intextsep{0pt}
    \setlength\columnsep{1em}
    \begin{wrapfigure}{r}{0.2\linewidth}
        \includegraphics[width=\linewidth]{\images/clone-token.png}\vspace*{1em}
        \includegraphics[width=\linewidth]{\images/clone-token-2.png}
        \centering\textbf{\scriptsize\color{darkcandyapplered}Clone \mbox{Tokens}\\}
    \end{wrapfigure}
    When you cast the \textbf{Clone Spell}, choose an allied unit (depending on the \svg{empower} used) and an adjacent empty space on the Combat Board to that Unit.
    Then place the small Clone Token on the chosen unit and the bigger Token of the same color on the empty space.
    The token on the empty space represents the Clone.
    It has the same statistics and special abilities as the original Unit, but its \svg{health_points} is only 1 and its \svg{defense} is 0.
    Additionally, its defense and health cannot be increased by any effects from Cards or other units. %what is with other statistics?
    If the Clone is attacked (even without taking damage) or takes even 1 \svg{damage} from a Spell, it perishes. %clarification?
    If the original Unit is removed from the Combat Board, remove its Clone as well.
    The Clone doesn't inherit any effects that have been played on the original Unit.
    %in official cove book is a example. is it necessary?.
\end{expansion}
\end{multicols}

\vspace*{1em}
\begin{expansion}{conflux}
    \begin{multicols}{2}
    \subsection*{Summon Elemental Unit Cards and Tokens}
    The \textbf{Summon Elemental Spells} allow you to add a Elemental Unit to the Combat Board during Combat.
    Keep the Summon Unit Card Deck face up near the neutral Unit Decks.
    Depending on the \svg{empower} used, you can Summon either a non-Reinforced Unit (a Few) or a Reinforced One (a Pack).
    Take the corresponding Unit Card from the deck of the Summoned Units and place it on a chosen empty space.
    You can decide to place the \textbf{Summon Unit Tokens} instead of the Cards.
    Using the battlefield Board you \textbf{must} place the Token.
    Summoned Units remain under your control until they perish or the Combat ends, whichever comes first.
    A Summoned Unit is not added to your Unit Deck.
    \columnbreak
        \begin{multicols}{2}
        \centering
        \includegraphics[width=0.9\linewidth]{\cards/unit-summoned-few.png}\par
        \textbf{\scriptsize\color{darkcandyapplered}A Few Fire Elementals}\par
        \smallskip
        \includegraphics[width=0.5\linewidth]{\images/summon-token-fire.png}\par
        \centering\textbf{\scriptsize\color{darkcandyapplered}Fire Elemental Token}\par
        \smallskip
        \includegraphics[width=0.5\linewidth]{\images/summon-token-earth.png}\par
        \centering\textbf{\scriptsize\color{darkcandyapplered}Earth Elemental Token}\par

        \columnbreak
        \includegraphics[width=0.9\linewidth]{\cards/unit-summoned-pack.png}\par
        \textbf{\scriptsize\color{darkcandyapplered}A Pack Fire Elementals\\}\par
        \smallskip
        \includegraphics[width=0.5\linewidth]{\images/summon-token-water.png}\par
        \centering\textbf{\scriptsize\color{darkcandyapplered}Water Elemental Token\\}\par
        \smallskip
        \includegraphics[width=0.5\linewidth]{\images/summon-token-air.png}\par
        \centering\textbf{\scriptsize\color{darkcandyapplered}Air Elemental Token}\par
        \end{multicols}
    \end{multicols}
\end{expansion}

\begin{multicols*}{2}

\subsection*{New Unit Abilities}
The following components or keywords belong to new Unit abilities. The Tokens are usually placed on other Units and influence their statistics.


\begin{expansion}{stronghold}
	\subsection*{Attack Tokens}
	\setlength\intextsep{0pt}
    \setlength\columnsep{1em}
    \begin{wrapfigure}{r}{0.2\linewidth}
        \includegraphics[width=\linewidth]{\images/attack-token.png}
        \centering\textbf{\scriptsize\color{darkcandyapplered}Attack \mbox{Token}\\}
    \end{wrapfigure}
    The \textbf{Orge Unit} from the Stronghold Faction can use its \svg{unit_other} ability to place Attack Tokens on other allied Unit Cards.
    Units with such a token gains an additional +1 \svg{attack} or +2 \svg{attack}, as indicated by the Tokens side.
    Each Unit can have only one such Token at a time.
    If a Unit would gain a second one, the player controlling it chooses which one to keep.
\end{expansion}
\vspace*{1em}

\begin{expansion}{stronghold}
	\subsection*{Corrosion Tokens}
	\setlength\intextsep{0pt}
    \setlength\columnsep{1em}
    \begin{wrapfigure}{r}{0.2\linewidth}
        \includegraphics[width=\linewidth]{\images/corrosion_token.png}
        \centering\textbf{\scriptsize\color{darkcandyapplered}Corrosion \mbox{Token}\\}
    \end{wrapfigure}The \textbf{Behemoth Unit} from the Stronghold Faction can use his \svg{unit_attack} ability to place a Corrosion Token on the defending units, which reduces their \svg{defense} by 1 to a minimum of "0". A unit can have only one such Token at a time.
	If a Unit should gain a second Token, discard one of the Tokens. Unless removed by Spells or other effects, a Corrosion Token stays on a Unit until end of combat.
\end{expansion}

\vspace*{1em}
\begin{expansion}{cove}
	\subsection*{Weakness Tokens}
  \setlength\intextsep{0pt}
  \setlength\columnsep{1em}
  \begin{wrapfigure}{r}{0.2\linewidth}
		\includegraphics[width=\linewidth]{\images/weakness-token.png}
        \centering\textbf{\scriptsize\color{darkcandyapplered}Weakness \mbox{Token}\\}
	\end{wrapfigure}
  The \textbf{Few Sorceresses Unit} from the Cove Faction can use its \svg{unit_other} abilty to place a Weakness Token on any Unit.
  A \textbf{Pack Sorceresses} deal that Token automatically to their defending Unit because of their \svg{unit_attack} ability.
  A unit with this Token suffers -1 \svg{attack} or -2 \svg{attack}, as indicated by the Token's side.
  Each Unit can have only one such Token at a time.
  If a Unit would gain a second one, the player controlling it chooses which one to keep.
\end{expansion}

\begin{expansion}{inferno}
	\subsection*{Summon Daemons}
	The \textbf{Pit Lords Unit} from the Inferno Faction may Summon Daemon Units\index{Summoning} during Combat.
	This effect cannot Summon Units from the Neutral Units Deck.
	Place the summoned Unit adjacent to the summoning Unit.
	Summoned Units Activate in the Round they were summoned if their Initiative is lower or equal to the Initiative of the currently Activated Unit.
	Otherwise, treat them as if they already activated this Combat Round.
	After Combat, unless stated otherwise, the Summoned Units are added to your Unit Deck.
\end{expansion}

\vspace*{1em}
\begin{expansion}{stretchgoals2}
	\subsection*{Alternative Manticore Card}
	%\setlength\intextsep{0pt}
    %\setlength\columnsep{1em}
    %\begin{wrapfigure}{r}{0.3\linewidth}
        %\includegraphics[width=\linewidth]{\boxcover/stretch_goals_faction.jpg}
    %\end{wrapfigure}
    There is a second, but slightly different copy of a \textbf{Manticore Unit Card} in the game. When you play the Dungeon Faction, you can choose one of these Cards to play with in the current Scenario.
\end{expansion}

\vspace*{1em}
\subsection*{You have still questions?}
Many Cards in \textbf{Heroes of Might and Magic 3: The Board Game} have ambiguous or misleading wordings.
Fortunately the developers of the game have clarified a lot of Card effects.\par
\begin{minipage}{5.7cm}
    The Community creates a \href{https://homm3bg.wiki/}{\textbf{Fanmade Database}}: a collection of nearly all components of the game with tons of notes for a better understanding of specific rules. If you have questions about single Cards, visit this Website and let you help.
\end{minipage}
\hfill
\begin{minipage}{2cm}
    \begin{center}
        \includegraphics[width=\linewidth]{\qr/wiki-database.png}
        \scriptsize Link to the \textbf{Wiki}
    \end{center}
\end{minipage}\par

\end{multicols*}
