% !TeX spellcheck = en_US
\addsection{Trading}{\skills/armorer.png}

\iftoggle{printable}{\vspace{-\baselineskip}}{}

\begin{multicols}{2}

The \pagelink{Trading Post}{Trading Post Field} and other effects allow you to either:
\begin{itemize}
  \item \pagetarget{Trading}{trade}\index{Trading} multiple Resources with the game in accordance to the \hyperlink{Trade Table}{table at the back cover},
  \item \textbf{remove exactly one card} from your hand at the trading post to gain 1 \svg{gold},
    \note{5}{Specialty, Statistic, Starting Ability and Magic Arrows \textbf{cannot be removed} in the Trading Post.}\par
  \item or buy a \pagelink{War Machines}{War Machine}, if you have the Rampart expansion.
\end{itemize}
In \textbf{Alliance} and \textbf{Cooperative} Scenarios, players are allowed to trade Resources and cards following these rules:
\begin{itemize}
  \item In Alliance Scenarios, allies may trade Resources freely at any time on their Turns except during Combat.
  \columnbreak
  \item In Cooperative Scenarios, Resources may be given to other players when Visiting a Trading Post. This works \textbf{in addition} to the regular effect of the Trading Post during a single Visit.
  \item In both Scenario types, allies may trade \textbf{Spell} and \textbf{Artifact} cards in any mix if they have heroes on adjacent Fields.
    Only \textbf{cards from their hands} may be traded and you must give and receive an equal amount of cards.
\end{itemize}

\medskip

\begin{center}
  \includegraphics[width=0.7\linewidth]{\art/ammo_cart.png}
\end{center}

\end{multicols}

\begin{scaledfigure}[blanker]
  \centering
  \includegraphics[width=\linewidth, height=\myspace, keepaspectratio]{\art/gold_dragon.jpg}
\end{scaledfigure}
