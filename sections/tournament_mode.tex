\addsection{Tournament Mode}{\spells/forgetfulness.png}

\begin{multicols*}{2}

The Tournament Mode has some rule changes in order to remove some randomness from the game to allow more tactical decisions and to give players more control.

\subsection*{Changes during setup}

Do following things during setup:
\begin{itemize}
    \item Remove the Ability Card "Diplomacy" and the Artifact Card "Hourglass of the Evil Hour" from the play.
    \item Both players roll 2 \svg{resource_die}. The player with the highest amount of resources on their die will get to choose who starts and who will go second.
    \item The second player gains \svg{morale_positive} at the start of the game.
\end{itemize}
\subsection*{Building the Map}

The Map in a Tournament Scenario is build by the players and it's \textbf{completely revealed} at the start of the game.
The Scenario's Map layout only shows, how the center of the Map should look like and which Tiles are needed to build the rest of the Map.
The placement decisions of all other Map Tiles assigned to the Scenario is left two both players.
While placing new Tiles, the \pagelink{Placing}{general rules}
Follow these rules to build the Map and \textbf{place all Map Tiles always face up}:
\begin{enumerate}
    \item The starting player discovers the Map Tiles in the center of the map and rotate them freely.
    \item Starting with the second player, both players take turns adding all, if any, of the Near Map Tiles to the scenario Map, one at a time.
    \begin{itemize}
        \item If possible, a Near Map Tile must be placed next to a Center Map Tile.
        \item At least 2 Near Map Tiles in every Scenario must have an Obelisk location on them.
    \end{itemize}
    \item Once all Near Map Tiles have been placed, the first player must place their starting Tile adjacent to at least one near Map Tile. The other player must place their Starting Map Tile at the opposite side of the Scenario Map - one the furthermost position away from the first player's Starting Map Tile (counted in Tiles).
    \item Starting with the First player, both players take turns adding their Far Map Tiles to the Scenario Map, one at a time
    \begin{itemize}
        \item Every player's first Far Map Tile must be placed adjacent to their Starting Map Tile.
        \item The remaining Map tiles must follow the \pagelink{Placing}{standard Tile Placement rules}.
    \end{itemize}
\end{enumerate}

\subsection*{Additional rules during the game}

The following rules apply to every tournament scenario:
\begin{itemize}
    \item At the Start of the first Round of a Scenario, \textbf{each player can once reshuffle} their hand of cards back into their Deck to draw a new starting hand.
    \item All players can use morale tokens for a \textbf{new morale action}: During any \textbf{Search} action (regardless of the type of the cards), discard all drawn cards and perform the Search action again.
    \item If you Remove Artifact Cards during the game, \textbf{keep them beside your Deck} as a reminder until end of game, when your pile of artifact Cards counts toward your final Victory Points score.
    \item
\end{itemize}

\subsection*{Scoring Victory Points}
When either the scenario reaches its round limit or any player completes the Scenario's victory condition, the game ends and both players count up their Victory Points (VP).
The player with the most VP wins the Scenario.

%I suggest to create a table for victory Points. Something like this one...
\includegraphics[width=\linewidth]{\tables/test_table.png}

%\begin{itemize}
    %\item Score \textbf{1 VP} for...
    %\begin{itemize}
        %\item ...every controlled mine or settlement,
        %\item ...every Building in your Town,
        %\item ...every 2 Artifact Cards in your Deck,
        %\item ...each of your Main Hero's Level.
    %\end{itemize}
    %\item Score \textbf{2 VP} for defeating the Enemy's Secondary Hero.
    %\item Score \textbf{4 VP} for defeating the Enemy's Main Hero (once per Scenario).
    %\item Score \textbf {X VP} for additional objectives outlined in a given Scenario's description.
%\end{itemize}


\end{multicols*}
