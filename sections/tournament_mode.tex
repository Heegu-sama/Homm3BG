\addsection{Tournament Mode}{\images/homm-shield.png}\pagetarget{Tournament Mode}{}

\begin{multicols}{2}

The 1 vs. 1 Tournament Mode has some rule changes in order to remove some randomness from the game to allow more tactical decisions and to give players more control.

\textbf{Note:} The Tournament Mode is designed only with the Core Game in mind. Rules to include expansion content (e.~g. \pagelink{Sea Map Tiles}{Sea Map Tiles}) are not known.

\subsection*{Changes during Setup}

Do following things during setup:
\begin{itemize}
    \item Remove the Ability Card ``Diplomacy'' and the Artifact Card ``Hourglass of the Evil Hour'' from the game before building their belonging decks.
    \item Both players roll 2 \svg{resource_die}. The player with the highest amount of resources on their dice will get to choose who starts and who will go second.
    \item The second player gains \svg{morale_positive} at the start of the game.
\end{itemize}


\subsection*{Building the Map}

The Map in a Tournament Scenario is build by the players and it will be \textbf{completely revealed} at the start of the game.
The Scenario's Map Layout only shows how the center of the Map should look like and which Tiles are needed to build the rest of the Map.
The placement decisions of all other Map Tiles assigned to the Scenario is left to both players.\par

Follow these rules to build the Map and \textbf{place all Map Tiles always face up}:
\begin{enumerate}
    \item The starting player places the Map Tiles in the center of the map, matching them to the Scenario's Map Layout, and discovers and rotates them freely.
\columnbreak
    \item Starting with the second player, both players take turns adding all, if any, of the Scenario's Near Map Tiles to the Map, one at a time.
    \begin{itemize}
        \item If possible, a Near Map Tiles must be placed next to a Center Map Tile.
        \item At least 2 Near Map Tiles in every Scenario must have an Obelisk location on them.
        \item The pool of available Near Map Tiles is facedown, the player draws a random Tile, reveals it, and chooses how to place it.
    \end{itemize}
    \item Once all Near Map Tiles have been placed, the first player must place their Starting Map Tile adjacent to at least one Near Map Tile.
          The other player must place their Starting Map Tile at the opposite side of the Map - on the furthermost position away from the first player's Starting Map Tile (counted in Tiles).
    \item Starting with the first player, both players take turns adding the Scenario's Far Map Tiles to the Map, one at a time.
    \begin{itemize}
        \item Every player's first Far Map Tile must be placed adjacent to their Starting Map Tile.
        \item The remaining Map tiles must follow the \pagelink{Placing}{standard Tile Placement rules}.
        \item The pool of available Far Map Tiles is facedown, the player draws a random Tile, reveals it, and chooses how to place it.
    \end{itemize}
\end{enumerate}

\subsection*{Additional rules during the game}

The following rules apply to every tournament scenario:
\begin{itemize}
    \item At the Start of the first Round of a Scenario, \textbf{each player can once reshuffle} their hand of cards back into their Deck to draw a new starting hand of Cards.
    \item All players can use Morale Tokens for a \textbf{new Morale Action}: During any \textbf{Search} action (regardless with which type of Cards), discard all drawn Cards and perform the Search action again.
    \item If you Remove Artifact Cards during the game, \textbf{keep them beside your Deck} as a reminder until end of game, when your pile of Artifact Cards counts toward your final Victory Points score.
    \item If Visiting a \textbf{Redwood Observatory Map Location} in Tournament Scenarios use its \pagelink{Observatory}{alternative effect}.
    \item \textbf{Optional rule}: You can decide to play a variant with \pagelink{Split Artifact and Spell Decks}{split Artifact and Spell Decks} to have a bit more control what you draw.
\end{itemize}

\vspace*{\fill}
\columnbreak
\subsection*{Scoring Victory Points}
When either the scenario reaches its round limit or any player completes the Scenario's victory condition, the game ends and both players count up their Victory Points (VP) according to the table below.
The player with the most VP wins the Scenario.

\hommtable{20}{
    \begin{tabularx}{\linewidth}{>{\hsize=0.1\hsize\linewidth=\hsize}X
            >{\hsize=0.8\hsize\linewidth=\hsize}X
        }
        \darkcell[0.8]{VP} & \darkcell[0.8]{Category} \\
        \darkcell[0.6]{1} &
        \lightcell[0.6]{per controlled mine or settlement} \\
        \darkcell[0.6]{1} &
        \lightcell[0.6]{per building in your town} \\
        \darkcell[1.2]{1} &
        \lightcell[1.2]{per two artifact Cards in your Deck (including removed artifacts)} \\
        \darkcell[1.2]{1} &
        \lightcell[1.2]{per each of your\\ Main Hero's Level} \\
        \darkcell[1.2]{2} &
        \lightcell[1.2]{each time you defeat a Secondary Hero} \\
        \darkcell[1.2]{4} &
        \lightcell[1.2]{if you defeat the enemy's Main Hero (only once)} \\
        \darkcell[1.2]{X} &
        \lightcell[1.2]{per additional objectives outlined in a given Scenario's description} \\
    \end{tabularx}
}
\end{multicols}

\vspace*{\fill}
\framedimage[\linewidth]{\art/harpy.jpg}
