% !TeX spellcheck = en_US
\addsection{Tournament Mode}{\images/homm-shield.png}\pagetarget{Tournament Mode}{}

\begin{multicols}{2}

The 1 vs. 1 Tournament Mode has some rule changes in order to remove some randomness from the game to allow more tactical decisions and to give players more control.

\textbf{Note:} The Tournament Mode is designed only with the Core Game in mind. Rules to include expansion content (e.~g. \pagelink{Sea Map Tiles}{Sea Map Tiles}) are not known.

\subsection*{Changes during Setup}

Before starting the game, do the following:
\begin{itemize}
    \item Remove the ``Diplomacy'' Ability Card and the ``Hourglass of the Evil Hour'' Artifact Card from the game before building their respective decks.
    \item Both players roll 2 \svg{resource_die}. The player with the highest amount of resources on their dice chooses who starts.
    \item The second player gains \svg{morale_positive} at the start of the game.
\end{itemize}

\subsection*{Building the Map}

The Map in a Tournament Scenario is built by the players and is \textbf{completely revealed} at the start of the game.
The Scenario's Map Layout only shows how the center of the Map should look like and which Tiles are needed to build the rest of the Map.
The placement decisions of all other Map Tiles assigned to the Scenario are left to both players.\par

Follow these rules to build the Map and \textbf{place all Map Tiles always face up}:
\begin{enumerate}
    \item The starting player places the Map Tiles in the center of the map, matching them to the Scenario's Map Layout, and discovers and rotates them freely.
\columnbreak
    \item Starting with the second player, both players take turns adding all, if any, of the Scenario's Near Map Tiles to the Map, one at a time.
    \begin{itemize}
        \item If possible, Near Map Tiles must be placed next to a Center Map Tile.
        \item At least 2 Near Map Tiles in every Scenario must have an Obelisk location on them.
        \item The pool of available Near Map Tiles is facedown. Players draw a random Tile, reveal it, and choose how to place it.
    \end{itemize}
    \item Once all Near Map Tiles have been placed, the first player must place their Starting Map Tile adjacent to at least one Near Map Tile.
          The other player must place their Starting Map Tile at the opposite side of the Map - on the furthermost position away from the first player's Starting Map Tile (counted in Tiles).
    \item Starting with the first player, both players take turns adding the Scenario's Far Map Tiles to the Map, one at a time.
    \begin{itemize}
        \item Every player's first Far Map Tile must be placed adjacent to their Starting Map Tile.
        \item The remaining Map tiles must follow the \pagelink{Placing}{standard Tile Placement rules}.
        \item The pool of available Far Map Tiles is facedown. Players draw a random Tile, reveal it, and choose how to place it.
    \end{itemize}
\end{enumerate}

\subsection*{Additional rules during the game}

The following rules apply to every tournament scenario:
\begin{itemize}
  \item \textbf{Round One Mulligan:} At the start of the \nth{1} Round, each player may reshuffle their hand of Cards back into their Deck of Might and Magic and draw a new starting hand of Cards once.
  \item All players can use Morale Tokens for a \textbf{new Morale Action}: During \textbf{any Search action} discard all drawn Cards and perform the Search action again.
  \item If you remove Artifact Cards during the game, \textbf{keep them beside your Deck} until the end of the game, when your pile of Artifact Cards counts toward your final Victory Points score.
  \item When Visiting a \textbf{Redwood Observatory} in Tournament Scenarios, use its \pagelink{Observatory}{alternative effect}.
  \item \textbf{Optional rule}: You can play with \pagelink{Split Artifact and Spell Decks}{split Artifact and Spell Decks} variant to have more control over the Cards you draw.
\end{itemize}

\vspace*{\fill}
\begin{center}
  \transparent{0.2}\includegraphics[width=1.05\linewidth]{\art/berserk.png}
\end{center}
\vspace*{\fill}
\columnbreak
\subheader{Scoring Victory Points}
When either the scenario reaches its round limit or any player completes the Scenario's victory condition, the game ends and both players count up their Victory Points (VPs) according to the table below.
The player with the most VPs wins the Scenario.

\bigskip

\begin{table}[H]
  \hommtable{26}{
    \centering
    \medskip
    \textbf{Victory Point scores}\\
    \bigskip
    \begin{tabularx}{0.95\linewidth}{>{\hsize=0.12\hsize\linewidth=\hsize}X
        >{\hsize=0.8\hsize\linewidth=\hsize}X
      }
      \darkcell[0.8]{VPs} & \darkcell[0.8]{Category} \\
      \darkcell[1.2]{1} &
      \lightcell[1.2]{fore every controlled Mine or Settlement} \\
      \darkcell[1.2]{1} &
      \lightcell[1.2]{for every building in your Town} \\
      \darkcell[1.8]{1} &
      \lightcell[1.8]{for every two artifact Cards in your Deck (including removed ones)} \\
      \darkcell[1.2]{1} &
      \lightcell[1.2]{for every Level of your\\ Main Hero} \\
      \darkcell[1.2]{2} &
      \lightcell[1.2]{for every \mbox{Secondary} Hero defeated} \\
      \darkcell[1.2]{4} &
      \lightcell[1.2]{for defeating the enemy's Main Hero (only once)} \\
      \darkcell[1.6]{X} &
      \lightcell[1.6]{for any additional objectives outlined in a given Scenario's description} \\
      \end{tabularx}
  }
\end{table}
\end{multicols}
