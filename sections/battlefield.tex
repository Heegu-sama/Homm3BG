% !TeX spellcheck = en_US
\pagetarget{Adventure and Skirmish Modes}{\addsection{Battlefield}{\spells/frenzy.png}}

\begin{expansion}{battlefield}
  \begin{multicols}{2}
  \subsection*{Overview}
  The Battlefield expansion introduces the large \pagelink{Battlefield Combat}{Battlefield Board}, which you can \textbf{optionally} use instead of Combat Board.

  \bigskip
  The expansion also introduces 2 similarly structured 1~vs.~1 game modes whose central idea is to engage in a tactical combat on the Battlefield Board.
  Neither of these 2 modes use \pagelink{Map}{Map} anymore.
  They differ in the way players build their Unit and Player Decks:
  \smallskip
  \begin{itemize}
    \item In \textbf{Skirmish Mode}, players receive pre-constructed Decks and armies, over whose composition they have little influence.
      With their Decks the players engage in Combat immediately.
    \item In \textbf{Adventure Mode}, a card-driven Adventure Phase is played before the final Combat, which replaces the Map discovery from regular games.
      In this phase, players use Adventure cards to gather Resources, expand their Town, and develop their Heroes, thus deciding for themselves which units to recruit and which types of cards to include in their Deck, rather than being limited to pre-constructed sets.
  \end{itemize}

  \begin{center}
    {\transparent{0.2}{\includegraphics[width=0.8\linewidth]{\art/quiet_eye_of_the_dragon.png}}}
  \end{center}\par

  \vspace*{\fill}
  \columnbreak

  \subsection*{Victory Condition}
  To achieve victory in Skirmish or Adventure, the player must defeat all enemy units in the final Combat.

  \subsection*{Rule Ambiguity}
  \imagecaption{Important Note:}
  Both game modes were developed with only the \textbf{Core Game in mind}.
  Therefore, official rules leave many questions unanswered, especially (but not exclusively) concerning card effects from expansions.
  If you don't want to remove all Expansion cards from the game, you'll have no choice but to consult house rules.

  \smallskip
  The following note box provides some clues about missing clarifications from the developers.
  But when learning the rules you should skip this part for now.

  \smallskip
  \note{18}{Open Questions in Skirmish and/or Adventure Mode:
  \begin{itemize}
    \item Which else cards should be removed from the game (beside Diplomacy and Logistics)?
    \item Which cards should be removed in Skirmish (cause there are only Combat effects useful)?
    \item How to handle cards with Permanent effects in Adventure Mode ?
    \item How to handle cards with a ``Remove'' Effect (e.~g. Expert Learning) in Adventure Mode?
    \item How to handle cards with both a Combat and an Adventure Effect?
  \end{itemize}
  }\par

  \vspace*{\fill}
  \end{multicols}
\end{expansion}

\newpage
\begin{expansion}{battlefield}
  \subheader{Skirmish and Adventure Setup}
  \begin{multicols}{2}
    \subsection*{Common Setup Steps \raisebox{.3\height}{{\scriptsize\encirclelegend{goblin}{C}}}}
    \begin{enumerate}[leftmargin=1.5em]
      \item Place the Battlefield Board in the middle of the table.
      \item Each player chooses a Faction and a Hero of that Faction and takes the following components:
      \begin{itemize}
        \item[a)] Hero Board
        \item[b)] 3 Hero-specific Specialty cards and 1 Hero-specific Ability card
        \item[c)] 7 unit cards and 7 unit miniatures
        \item[d)] 20 Faction Cubes
      \end{itemize}
      \item Prepare the Artifact, Spell, and Ability decks normally.
        Remove the following Ability cards from the play: ``Diplomacy'', ``Logistics''.
      \item Prepare the Positive and Negative \pagelink{Morale Cards}{\textbf{Morale cards}} (including all cards with the battlefield symbol).
      \item Prepare the following Tokens near the Battlefield Board: Black Cubes, Damage Tokens, Defense/Paralysis Tokens, Obstacles, and the Initiative Token.
      \item Prepare the Attack and Resource Dice.
      \item Build your starting Deck as as per \pagelink{Setup Starting Deck}{steps 11--13 of Setup}.
      \item Both players roll 2 \svg{resource_die}.
        The player with the highest total is the first player.
        Use the Initiative Token to track the current first player.
     \end{enumerate}
     \columnbreak

     \subsection*{Additional Steps for Adventure Mode \raisebox{.3\height}{{\scriptsize\encirclelegend{darkcandyapplered}{A}}}}
     \begin{enumerate}[leftmargin=1.5em, resume]
      \item Prepare Resource Tokens and Treasure Dice.
      \item Each player takes a Town Board with the following starting setup:
      \begin{itemize}
        \item[a)] Set income tracks to 10 \svg{gold}, 0 \svg{building_materials}, and 0 \svg{valuablegreater}.
        \item[b)] No Town Buildings or units in the Unit Deck.  % Clarification: https://boardgamegeek.com/thread/3284421/article/44125381#44125381
      \end{itemize}
      \item Starting resources: 15 \svg{gold}, 3 \svg{building_materials}, and 1 \svg{valuablegreater}.
      \item Shuffle the Adventure Deck and place it face down.
      \item Place the Round Tracker and put a Black Cube on Round 1.
      \item Set Hero Level to I.
      \item Set your starting Deck aside for now; you will only need it for the final Combat.
  \end{enumerate}
  \vspace*{\fill}
\end{multicols}

\begin{center}
  \begin{scriptsize}
  \begin{tikzpicture}
    \draw (0, 0) node[inner sep=0] {\makebox[\textwidth][c]{\includegraphics[width=0.9\linewidth]{\images/battlefield_setup.png}}};
    \draw (-4, 0) node {\encircle{1}};
    \draw (-5, -1.5) node {\encircle{2a}};
    \draw (3.8, 2.9) node {\encircle{2a}};
    \draw (-5.5, 1.2) node {\encircle{2c}};
    \draw (2, 0.2) node {\encircle{2c}};
    \draw (5.5, 1.2) node {\encircle{2c}};
    \draw (5.3, -2) node {\encircle{3}};
    \draw (2.9, -1.6) node {\encircle{4}};
    \draw (-2, 2.8) node {\encircle{5}};
    \draw (2.8, -2.5) node {\encircle{6}};
    \draw (0, -2.5) node {\encircle{7}};
    \draw (0, 3.4) node {\encircle{7}};
    \draw (3.5, -3) node {\encirclered{9}};
    \draw (-3.8, 3) node {\encirclered{9}};
    \draw (-2, -2) node {\encirclered{10}};
    \draw (2, 3) node {\encirclered{10}};
    \draw (-6, -1.7) node {\encirclered{11}};
    \draw (5, 3) node {\encirclered{11}};
    \draw (1, -1.6) node {\encirclered{12}};
    \draw (-1.9, 3.5) node {\encirclered{13}};
  \end{tikzpicture}
  \end{scriptsize}
\end{center}
\end{expansion}

\newpage
\begin{expansion}{battlefield}
  \pagetarget{Skirmish}{\subheader{Skirmish Mode}}
  \begin{multicols}{2}
    Skirmish Mode has no Adventure Phase.
    Skip that phase, prepare your Deck, and proceed directly to the final Combat.\par
    \vspace*{1em}
    \subsection*{Deck Preparation}
    Both players decide on a game length: \textbf{short, medium, or large}.
    Depending on the chosen game length, both players modify their Hero's Level, their Deck (by drawing additional cards), and their Unit Deck as shown in the table below.\par

    \vspace*{1em}
    \columnbreak
    \subsection*{Starting the final Combat}
    After modifying your Deck, shuffle it and draw cards up to your Hand Limit.
    The first player gets the Initiative Token.
    Then follow the rules for \pagelink{Battlefield Combat}{Combat on the Battlefield Board}.

    \vspace*{\fill}
  \end{multicols}

    \hommtable{26}{
      \begin{tabularx}{0.98\linewidth}{>{\hsize=0.2\hsize\linewidth=\hsize}X
          >{\hsize=0.25\hsize\linewidth=\hsize}X
          >{\hsize=0.25\hsize\linewidth=\hsize}X
          >{\hsize=0.25\hsize\linewidth=\hsize}X
        }
        & \darkcell{\raisebox{-1.3\height}{Short Game}} & \darkcell{Medium Game} & \darkcell{\raisebox{-1.3\height}{Large Game}} \\
        \darkcell[1.2]{Hero Level} &
        \lightcell[1.2]{Level III} &
        \lightcell[1.2]{Level V} &
        \lightcell[1.2]{Level VII} \\
        \darkcell[1.2]{Specialty cards} &
        \lightcell[1.2]{Level I Specialty\\card} &
        \lightcell[1.2]{Level I and Level~IV Specialty cards} &
        \lightcell[1.2]{All Specialty cards} \\
        \darkcell[1.8]{Ability cards} &
        \lightcell[1.8]{Draw 3 Ability cards and keep 2.} &
        \lightcell[1.8]{Draw 4 Ability cards and keep 3.} &
        \lightcell[1.8]{Draw 6 Ability cards and keep 4.} \\
        \darkcell[1.8]{Artifact cards} &
        \lightcell[1.8]{Draw 2 Artifact cards and keep 1.} &
        \lightcell[1.8]{Draw 3 Artifact cards and keep 2.} &
        \lightcell[1.8]{Draw 4 Artifact cards and keep 3.} \\
        \darkcell[2.2]{Spell cards} &
        \lightcell[2.2]{Draw 1 Spell card for a \svg[12]{might-yellow} Hero or 2 Spell cards for a \svg{magic-yellow} Hero.} &
        \lightcell[2.2]{Draw 2 Spell cards for a \svg[12]{might-yellow} Hero or 4 Spell cards for a \svg{magic-yellow} Hero.} &
        \lightcell[2.2]{Draw 3 Spell cards for a \svg[12]{might-yellow} Hero or 5 Spell cards for a \svg{magic-yellow} Hero.} \\
        \darkcell[2.2]{Units} &
        \lightcell[2.2]{Use all \svg{bronze} and \svg{silver} units on the ``Few'' side.} &
        \lightcell[2.2]{Use all \svg{bronze} and \svg{silver} units on the ``Pack'' side.} &
        \lightcell[2.2]{Use all units (\svg{bronze}, \svg{silver}, and \svg{golden}) on the ``Pack'' side.} \\
      \end{tabularx}
    }
  \begin{center}
    \vspace*{1em}
    \transparent{0.2}\includegraphics[width=0.35\linewidth]{\art/corpse.png}
  \end{center}
\end{expansion}

\newpage
\begin{expansion}[before=\vspace*{-11mm}]{battlefield}
  \pagetarget{Adventure}{\subheader{Adventure Mode}}
  \begin{multicols}{2}
  Adventure Mode is divided into two phases: the \textbf{Adventure Phase} and the \textbf{Final Combat}.
  The latter works identically to Skirmish Mode.\par
  Before starting the game, both players choose a game length by setting the number of Rounds in the Adventure Phase:
  \begin{itemize}
    \item \textbf{Short} game: 9 Rounds.
    \item \textbf{Medium} game: 13 Rounds.
    \item \textbf{Large} game: 16 Rounds.
  \end{itemize}
  \subsection*{Adventure Phase}
  The Adventure Phase lasts a certain number of Rounds, depending on the chosen game length.
  Follow these steps in order during each Round.\par
  \begin{enumerate}[leftmargin=1.3em]
    \item \textbf{Move the Round Tracker} -- skip this step in the first Round.
    Depending on the type of Round, resolve the following effects:
    \begin{itemize}
      \item In \textbf{Resource Rounds} both players get income normally.
      \item In \textbf{Astrologers Rounds} -- instead of drawing Astrologers Proclaim cards, players increase one of their income tracks (\svg{gold}, \svg{building_materials}, or \svg{valuablegreater}) in one of the following ways:
      \begin{itemize}
        \item Increase your preferred income track.
        \item Determine it randomly by rolling a \svg{resource_die}.
          If you do, gain the basic income of that resource immediately, which is 5 \svg{gold}, 2 \svg{building_materials}, or 1 \svg{valuablegreater}.
      \end{itemize}
    \end{itemize}
    \item \textbf{Draw Adventure Cards} -- skip this step in the last Round.
    The starting player draws 3 Adventure cards, keeps 1, and passes the remaining 2 to the other player, who keeps 1 and discards the other.\par
    \textbf{Note:} Only the starting player may use \svg{movement} effects to discard any number of the drawn Adventure cards and draw that many cards.\par
    \item \textbf{Player Turns} -- starting with the first player, both players take their turn by performing any of the following actions in any order:
    \begin{itemize}
      \item \textbf{Resolve} the chosen Adventure card, \textbf{discard} it, or \textbf{save it} for later turns.
        If you save it, place it face down on your Hero Board.
        You may only save 1 Adventure card at a time.
        If you want to save a new Adventure card while already having one saved, you must first resolve or discard the previously saved card.
      \item Resolve a previously saved Adventure card.
        You may only resolve \textbf{one Adventure card per turn}.
      \item Perform Town or Morale actions, or resolve your Hero's level ups, in the same way as in regular games.  % no-check-caps
      \item Use the effects of \pagelink{Set aside Cards}{set aside cards} once.
        Remove the card from the game after use.
    \end{itemize}\par
    \item \textbf{Change First Player} -- Pass the Initiative Token to the other player. Odd-numbered Rounds are started by the first player, and even-numbered Rounds by the second player.\par
  \end{enumerate}\par
  \smallskip
  In the last Round, players do not draw Adventure cards, but they may still resolve a saved Adventure card.
  After the last Round, the Adventure Phase ends and the final Combat begins.

  \subsection*{Starting the final Combat}
  Recover all spent \svg{expert}.
  Pick up your set-aside Deck, shuffle it, and draw cards up to your Hero's Hand \svg{hand} Limit.
  Remove all \svg{paralysis} tokens from your units, take all your unit miniatures, and begin Combat on the Battlefield Board following the \pagelink{Battlefield Combat}{Battlefield Combat rules}.
  \end{multicols}
\end{expansion}

\newpage
\begin{expansion}{battlefield}
  \begin{multicols}{2}
  \subsection*{Resolving Adventure Cards}
  There are two types of Adventure cards: Combat and Event.
  \medskip
  \begin{multicols*}{2}
    \footnotesize
    \begin{center}
      \begin{tikzpicture}
        \scriptsize
        \draw (0, 0) node[inner sep=0] {\includegraphics[width=1.1\linewidth]{\cards/adventure_event.png}};
        \draw (1.2, 1.9) node {\encircle{1}};
        \draw (-0.2, -0.2) node {\encircle{2}};
        \draw (-1, -1.2) node {\encircle{3}};
      \end{tikzpicture}
      \imagecaption{Adventure-Event card}
      \begin{enumerate}
        \item[\textbf{1.}] Name
        \item[\textbf{2.}] Type
        \item[\textbf{3.}] Effect
      \end{enumerate}
    \end{center}
    \columnbreak

    \begin{center}
      \begin{tikzpicture}
        \scriptsize
        \draw (0, 0) node[inner sep=0] {\includegraphics[width=1.1\linewidth]{\cards/adventure_combat.png}};
        \draw (1.2, 1.9) node {\encircle{1}};
        \draw (-0.2, 0.2) node {\encircle{2}};
        \draw (-1, -0.8) node {\encircle{4}};
        \draw (1, -0.8) node {\encircle{5}};
        \draw (0.4, -2.2) node {\encircle{6}};
      \end{tikzpicture}
      \imagecaption{Adventure-Combat card}
      \begin{enumerate}
        \item[\textbf{4.}] Combat Power Levels
        \item[\textbf{5.}] Battle Reward
        \item[\textbf{6.}] Basic Reward
      \end{enumerate}
    \end{center}
  \end{multicols*}
  To resolve an \textbf{Adventure-Event card}, resolve its printed effect.
  To resolve a \textbf{Adventure-Combat card}, follow these steps:
  \begin{enumerate}[leftmargin=1.3em]
    \item \textbf{Declare a Combat Power Level:}\index{Combat Power}
      Choose one of the three available Combat Power Levels.
      You must reach at least that value to gain the full reward.
    \item \textbf{Determine your Combat Power:}
      Select 2 of your units and roll 1 Attack Die for each.
      If you have fewer than 2 units, select as many as you have.
      Sum all unit Attack values and all Attack Die results to determine your final Combat Power.
    \item \textbf{Get the Reward:}
      If your Combat Power is equal to or higher than your declared Combat Power Level, take both the Basic Reward and the Battle Reward.
      Otherwise, take only the Basic Reward and place a \svg{paralysis} token on one of your units that took part in this Combat.
      This unit is now stunned.
      If you choose a stunned unit for a future Combat, do not add its Attack value to your Combat Power; then remove the \svg{paralysis} token from that unit.
  \end{enumerate}
  \columnbreak

   After resolving an Adventure card, place it in your own discard pile.
   For example, the Adventure-Event card ``Obelisk'' grows stronger with each Obelisk card you have played previously.\par

  \subsection*{Additional Rules}
  \begin{itemize}
    \item Your \svg{expert} limit applies to the entire Adventure Phase.
      Recover all spent \svg{expert} at the start of the final Combat.
    \item Your Deck is not available during the Adventure Phase.
      Place all cards of your starting Deck face down near your Hero Board.
      They will be used in the final Combat.
    \item Each time an effect instructs you to \pagetarget{Set aside Cards}{gain new cards} (usually from leveling up or resolving Adventure cards), proceed as follows:
    \begin{itemize}
      \item Cards that are \textbf{useful only in Combat} go directly into your Deck and remain there until the final Combat.
        You can't use these cards during Adventure Phase.
        Spells always go into your Deck, since their alternative effect (+1 \svg{empower}) is always useful in Combat.
      \item Cards that \textbf{are not useful in Combat} because they provide Resources (e.g., ``Estates'') or \svg{movement}, or provide discounts (e.g., ``Legs of Legion'', ``Wisdom''), are \textbf{set aside} next to your Hero Board.
        Each of these cards may be used once, then removed from the game.
        These cards do not go into your Deck.
      \item If you are unsure how to categorize a card, note that clear rules do not exist for every individual case, and you may need to agree on a house rule with your opponent.
    \end{itemize}
  \end{itemize}
  \end{multicols}
\end{expansion}

\begin{expansion}[before=\vspace*{-11mm}]{battlefield}
  \pagetarget{Battlefield Combat}{\subheader{Battlefield Combat}}
  \begin{multicols*}{2}
  Playing on the Battlefield Board requires using \pagelink{Miniatures}{Miniatures} for Combat instead of unit cards.

  \subsection*{\pagetarget{BF Obstacles}{Obstacle Tokens}}
  You can place three token types on the Battlefield Board: Effect, Obstacle and Wall/\allowbreak Gate.
  Unless otherwise stated, these Tokens count and work as \pagelink{Combatterminology}{Combat Obstacles}.
  To place Effect Obstacles on the Battlefield Board you'll need the card which creates the effect, such as the Fire Wall Spell: use the Tokens and put the card beside the Battlefield as a reminder.
  Effect Obstacles may be entered by units, if the card text allows it.
  \begin{figure}[H]
    \centering
    \begin{subfigure}[b]{0.3\linewidth}
      \centering
      \includegraphics[width=0.8\linewidth]{\images/bf-obstacle-1.png}
      \caption{\imagecaption{Effect}}
    \end{subfigure}
    \begin{subfigure}[b]{0.3\linewidth}
      \centering
      \includegraphics[width=0.9\linewidth]{\images/bf-obstacle-2.png}
      \caption{\imagecaption{Obstacle}}
    \end{subfigure}
    \begin{subfigure}[b]{0.3\linewidth}
      \centering
      \includegraphics[width=0.9\linewidth]{\images/bf-obstacle-3.png}
      \caption{\imagecaption{Wall/Gate}}
    \end{subfigure}
  \end{figure}
  \vspace*{-0.5\baselineskip}
  \note{3}{
    All unit Miniatures also count as Obstacles.
  }
  % \columnbreak

  \vspace*{-1em}
  \subsection*{Combat Setup}
  \setlength\intextsep{0pt}
  \setlength\columnsep{1em}
  \begin{wrapfigure}{r}{0.4\linewidth}
    \centering
    \includegraphics[width=0.8\linewidth]{\images/initiative-bf.png}
    \begin{center}
      \caption{\imagecaption[c]{Initiative\\Token}}
    \end{center}
    \vspace{1em}
  \end{wrapfigure}
  The starting player receives the \pagetarget{Initiative Token}{\textbf{Initiative Token}}.
  Starting with the player with the Initiative Token, both players take turns choosing and placing one Obstacle Tokens on the Battlefield Board one-by-one until \textbf{all Obstacle Tokens (except Effects) are placed}.
  No Obstacle can be adjacent to another Obstacle or to any player's deployment zone.

  {\centering
    \begin{tikzpicture}
      \draw (0, 0) node {\framedimage[\linewidth]{\examples/bf-deploy.png}};
      \draw (3.2, 0) node[rotate=90, fill=black, fill opacity=0.19, text opacity=1]
        {\textbf{\liberation\LARGE\textcolor{white}{\MakeUppercase{Deployment Zone}}}};
    \end{tikzpicture}
  }\\
  \imagecaption{Obstacles and Miniatures placed correctly are marked with the green border, and incorrectly -- with the red one.}\par
  \vspace{1em}
    The player with the Initiative Token chooses a deployment zone and places one of their units in it, then players take turns placing their units \textbf{one by one} each in their zone.
    The 5-unit limit does not apply in Battlefield modes, and players may place all Faction units available to them.
    When the last unit is placed, the Combat begins.
  \end{multicols*}
\end{expansion}

\newpage
\begin{expansion}[before=\vspace*{-11mm}]{battlefield}
  \begin{multicols*}{2}
  \subsection*{Battlefield Combat Rules}
  \begin{itemize}
    \item \textbf{Unit Movement} -- Each unit's Movement is equal to its Initiative value: a unit of Initiative 8 can move up to 8 hexes.
    \item \textbf{Ranged Movement} -- Ranged \svg{unit_ranged} units can either move or attack, not both.
    \item \textbf{Combat Penalty} -- Ranged \svg{unit_ranged} units suffer a Combat Penalty when attacking an adjacent unit or a unit that is \textbf{8 or more} hexes away from them.

    \item \textbf{Initiative Tie} -- If two or more opposing units have the same Initiative value, the player with the Token activates one of their units first.
      Then, the players alternate in activating all of their units of that same Initiative value one by one.
      Once all units with that Initiative value were activated, pass the Initiative Token to the other player, who keeps it until an Initiative tie of opposing units happens again.

    \item \textbf{In \pagelink{Adventure and Skirmish Modes}{Adventure and Skirmish} only} -- At the end of each Combat Round, \textbf{regain one used \svg{expert}} up to your Hero's Limit, and \textbf{draw 2 cards from your Deck}.
  \end{itemize}

  \vspace*{\fill}

  \begin{center}
    \includegraphics[width=1.05\linewidth]{\art/dendroid.jpg}
  \end{center}
  \end{multicols*}
\end{expansion}

\begin{expansion}[before=\vspace*{-11mm}]{battlefield}
  \subheader{Battlefield Board in regular games}
  \begin{multicols*}{2}
    Although Battlefield is designed for \pagelink{Adventure and Skirmish Modes}{Adventure and Skirmish modes}, it is possible to use the Battlefield Board in other game modes, albeit with slight rule modifications.

    \subsection*{Changes to Combat Rules}

    \begin{itemize}
      \item The attacker receives the Initiative Token at the start of the Combat Setup.
      \item In player vs. player Combat, you may place up to 7 units.
        In Neutral Combat, the 5-unit limit still applies.
    \end{itemize}

    \subsection*{Siege with Walls and Gate}
    During a Siege against a Town with a Citadel, Walls and Gate Obstacles are placed as shown below.
    Place no other Obstacles.

    \hspace*{-0.5em}
    \begin{tikzpicture}
      \draw (0, 0) node {\framedimage[\linewidth]{\examples/bf-deploy-wall.png}};
      \draw (2.5, 0) node[rotate=90, fill=black, fill opacity=0.15, text opacity=1]
        {\textbf{\liberation\LARGE\textcolor{white}{\MakeUppercase{Deployment Zone}}}};
    \end{tikzpicture}
    \imagecaption{The red border marks incorrectly placed Obstacles.}
    \columnbreak

    During a Siege, defending units may enter the Gate Token \textbf{only at the two middle hexes} where the Gate is printed.
    Attacking the Gate Token is still possible on all four hexes.\par
    \bigskip
    \framedimage[\linewidth]{\examples/bf-gate-token.png}
    \textit{\footnotesize{}Example: Halberdier and Crusader (defenders) may enter the Gate Token only along the green arrows, but Minotaur and Harpies are still able to attack the Gate Token from any side.}

    \subsection*{Combat with Neutral Units}
    Although it is recommended to use the Battlefield Combat Board only for \textbf{Player vs.~Player Combats}, it is also possible to use it in Neutral Combats.
    If you decide to use the Battlefield Board in Combats with Neutral Units, ignore the Round Limit: at the end of each Combat Round you may freely decide to Retreat or to start another Combat Round with no need to spend \svg{movement}.
    Be aware, however, that this may significantly extend the playtime.
  \end{multicols*}
\end{expansion}
