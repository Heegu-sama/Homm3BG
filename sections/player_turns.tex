% !TeX spellcheck = en_US
\addsection{Player Turns}{\spells/dimension_door.png}

\begin{multicols*}{2}

At the start of your Turn, refresh your hand of cards following these steps:
\begin{itemize}
  \item Discard any number of cards from your hand.
If your current hand exceeds your Hand Limit \svg{hand}, you must discard down to match the Limit.
  \item You may then draw cards up to your Hand Limit.
  \item Resolve any ``at the beginning of your turn'' abilities after drawing.
\end{itemize}
Your current Hand Limit\index{Hand Limit} depends on your Main Hero's \pagelink{Level}{Level}.
The beginning of your Turn is the only time your Hand Limit is checked.\par
There are three types of Actions players may take: \textbf{Movement}, \textbf{Town}, and \textbf{Morale}.
Once all players have spent all their Movement Points and do not wish to use any further Town or Morale Actions, the current Round is over.

\vspace*{\fill}
\begin{center}
  \transparent{0.2}\includegraphics[width=0.8\linewidth]{\art/town_portal.png}
\end{center}

\note{10}{
  Whenever you are instructed to gain (additional) MP, sometimes represented by the symbol \svg{movement-yellow}, that MP persists for \textbf{only the Turn it was gained on}.
  In the unlikely situation that two allied Heroes are forced onto the same Field, you must use your next MP to move one of them away from that Field.
}

\columnbreak
\subheader{Movement Actions}
\pagetarget{Movement}{Movement Actions}\index{Movement Actions} are performed by spending Movement Points.
A player can use Movement Actions \textbf{only during their own Turn}.\par
For every 1 MP spent, you can perform one of the following Actions:
\begin{itemize}
  \item Move a Hero 1 Field in any direction.
  \item \pagelink{Categories}{Revisit} a Field where your Hero is in.
  \item \pagelink{Timelimit}{Continue Combat} against Neutral Units for 1 additional Combat Round.
  \item \pagelink{Placing}{Discover a face down Map Tile} if your Hero is on a Field next to that Tile.
  \item Place a new Map Tile from your pool of Far (II–III) Map Tiles.
\end{itemize}

\begin{center}
  \includesvg[width=0.6\linewidth]{\images/movement_tokens}\\
  \medskip
  \footnotesize\textit{An active and an inactive Movement Token.}
\end{center}

\bigskip

Mark the amount of MP you have used by flipping your Movement Tokens over to their brown, inactive side.
If a player has both a Main and a \pagelink{Secondary}{Secondary Hero}, track their MPs separately.
Heroes can spend MP in any order.\par
Allied Heroes can move through each other but cannot stop their movement in the same Field.
When you move through a Field with an allied Hero, do not \pagelink{Categories}{Visit} the Field that the allied Hero is standing on.\par

\clearpage

\pagetarget{Town Actions}{\subheader{Town Actions}\index{Town Actions}}
You can perform each of the Town Actions listed below \textbf{once per Round}.
These Actions can be performed at any point during any player's Turn, except during Combat or when your Town Action would interrupt another Action.
For example, players cannot draw Spell cards simultaneously using the Spell Book Token.\par
When a player announces that they are about to start Combat, you may react to it with any number of Town Actions before performing any of the steps of \pagelink{Combatsetup}{setting up Combat.}\par
After performing a Town Action, flip the respective Token on its inactive side on your Town Board.
You cannot use that Action again until the start of the next Round, when the Tokens are refreshed.
\begin{itemize}
  \item [{\includegraphics[height=1.5\baselineskip, valign=c]{\images/build.png}}] Build Token\index{Build Token}, used to expand your \pagelink{Town}{Town}.
  \item [{\includegraphics[height=1.5\baselineskip, valign=c]{\images/population.png}}] Population Token\index{Population Token}, used to recruit and reinforce \pagelink{Units}{units} or to recruit \pagelink{Secondary}{a Secondary Hero}.
  \item [{\includegraphics[height=1.5\baselineskip, valign=c]{\images/spells.png}}]Spell Book Token\index{Spell Book Token}, used to purchase \pagelink{spells}{Spells}. Requires the Mage Guild Town building to use.
\end{itemize}

\vspace*{\fill}
\makebox[\linewidth]{\transparent{0.2}\includegraphics[width=0.8\linewidth]{\art/shield.png}}
\vspace*{\fill}

\columnbreak
\subsection*{Example Player Turn}

\textit{Alice, who is playing the Hero Catherine the Knight, begins her Turn.
She has 3 cards in her hand from the previous Round, and decides to discard 2 of them before drawing cards from her Deck up to her \mbox{Hand Limit \svg{hand}.}
The current limit is 5, since her Main Hero is Level 3, so she draws 4 cards after discarding (see \pagelink{Level}{Level Effects}).}\par
\textit{She then spends her Build Token to construct the \svg{golden} Dwelling, and then her Population Token to \pagelink{Units}{recruit} the \svg{golden} unit Champions.
She can do this, as she had previously built the prerequisite lower Level Dwellings (\svg{bronze} and \svg{silver}) and has enough \pagelink{Resources}{Resources} to both Build the Dwelling and to recruit the Champions.}\par
\textit{Now prepared for an upcoming battle, she spends a Movement Point to move her Main Hero to an adjacent Field currently occupied by Sandro the Necromancer, an enemy Main Hero controlled by Bob.
As Alice announces her intent to start Combat, both players still have an opportunity to perform Town Actions.}\par

\includegraphics[width=\linewidth]{\examples/catherine_attacks_sandro.png}
\vspace*{\fill}
\columnbreak

\pagetarget{Morale}{\subheader{Morale Actions}}\index{Morale Actions}
Players can gain or lose Morale through various game effects.
When you gain Morale, take a Positive Morale Token\index{Positive Morale} \svg{morale_positive}.
You may only have one such Token.
If you are about to gain a second Token, you may immediately spend the first one before gaining the second.
A Positive Morale Token may be spent to perform any of the following Actions at any time:
\begin{itemize}
  \item Draw a card from your Deck.
  \item Discard any number of cards, then draw that many cards.
  \item Reroll any Die you have thrown.
\end{itemize}
If you lose Morale, discard a Positive Morale Token \svg{morale_positive} if you have one, otherwise gain a Negative Morale\index{Negative Morale} Token \svg{morale_negative}.
Inversely, gaining Positive Morale while you have a Negative Morale Token discards the Token.
If you would gain a second Negative Morale Token, you must instead \textbf{discard your hand of cards} the next time you end your Turn.\par

\vspace*{\fill}
\begin{center}
{\transparent{0.2}\includegraphics[width=\linewidth]{\art/morale_low.png}}
\end{center}
\vspace*{\fill}

\note{5}{The Necropolis \svg{necro-yellow} Faction ignores any Morale effects.
  They cannot ever gain or lose Morale for any reason.
}
\columnbreak

\begin{expansion}[before=\vspace*{-11mm}]{battlefield}
\subsection*{\pagetarget{Morale Cards}{Morale cards in regular games}}\index{Morale cards}

You can use Morale cards \textbf{instead} of Tokens.
During setup, remove all Morale cards with the Battlefield symbol.
Then shuffle the Positive and Negative Morale card Decks separately and place them face down near the Round Tracker.

\vspace{1em}
\begin{scriptsize}
  \begin{tikzpicture}
    \draw (-2, 0) node[inner sep=0] {\makebox[0.47\linewidth][c]{\includegraphics[width=0.47\linewidth]{\cards/morale-positive.png}}};
    \draw (2.4, 0) node[inner sep=0] {\makebox[0.47\linewidth][c]{\includegraphics[width=0.47\linewidth]{\cards/morale-negative-battlefield.png}}};
    \draw (-1.2, 2.2) node {\encircle{\phantom{.}1\phantom{.}}};
    \draw (3.2, 2.2) node {\encircle{\phantom{.}1\phantom{.}}};
    \draw (-3.0, -1.4) node {\encircle{\phantom{.}2\phantom{.}}};
    \draw (1.5, -1.0) node {\encircle{\phantom{.}2\phantom{.}}};
    \draw (2.0, -2.0) node {\encircle{\phantom{.}3\phantom{.}}};
  \end{tikzpicture}
\end{scriptsize}
\vspace{-1em}
\begin{multicols*}{2}
  \centering
  \footnotesize
  \imagecaption{Positive Morale card}

  \imagecaption{Negative Morale card}
\end{multicols*}
\begin{figure}[H]
  \centering
  \footnotesize
  \begin{subfigure}[t]{0.27\linewidth}
    \begin{itemize}
      \item[\textbf{1.}] Name
      \item[\textbf{2.}] Effect
    \end{itemize}
  \end{subfigure}
  \begin{subfigure}[t]{0.7\linewidth}
    \begin{itemize}
      \item[\textbf{3.}] Battlefield symbol\\{\scriptsize\textit{(Use such cards only in \pagelink{Adventure and Skirmish Modes}{Adventure and Skirmish} games)}}
    \end{itemize}
  \end{subfigure}
\end{figure}

Every time you gain Positive Morale, discard a Negative Morale card (if you have any); otherwise draw 1 Positive Morale card.
You may have \textbf{2 Positive Morale cards at a time}.
Don't draw a third card.
Every time you lose Morale, discard a Positive Morale card (if you have any); otherwise draw 1 Negative Morale card.
You may have an \textbf{unlimited number of Negative Morale cards}.\par
Place newly gained Morale cards face up near your Hero Board.
Positive Morale cards may be used at any moment you wish.
Negative Morale cards must be used the next time when the situation indicated by the card occurs.
\textbf{Put used cards at the bottom of their respective Decks}.
If you must draw a Morale card but the respective Morale Deck is empty, reshuffle its Discard Pile to form a new Deck.

Keep in mind that in Adventure and Skirmish Mode the rules for Morale cards differ a bit -- read the official Battlefield rule book.\par
\end{expansion}
\end{multicols*}
