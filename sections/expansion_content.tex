% !TeX spellcheck = en_US
\addsection{Expansion Content}{\skills/pathfinding.png}

Numerous expansions have been released for Heroes III: The Board Game, introducing new gameplay elements.
Most expansion rules are integrated directly into the relevant sections of this rulebook where they belong.
Below you'll find an overview of all expansion content.

\vspace*{1em}

\textbf{The following contents are added by many of the expansions:}
\begin{multicols*}{2}
\subsection*{New Map Locations}
Each expansion adds new map locations to the game, which your Heroes can explore.
They are explained in \pagelink{All Map Locations}{All Map Locations}.

\subsection*{\pagetarget{Miniatures}{Unit Miniatures}}
The expansions make your units appear even more alive. 
If you want to play your combats with miniatures, place and move them over the combat board instead of the cards. 
You can put miniatures on top of the cards or keep the cards next to the combat board forming an initiative bar for a better visualization of the order of activating units.\par
\vspace*{1em}
If you play with miniatures, there are a few rules to follow:
\begin{itemize}
	\item Discard a drawn neutral unit card in neutral combats and draw another card in its place, if you draw \textbf{the same neutral unit more than once} OR \textbf{any unit you already have} in your army OR \textbf{any unit from your faction}. 
	\item While you Recruit Neutral Units, you can Recruit neither any units from a faction that is controlled by a player nor any units that already in any player's army. 
	Discard that Unit card and draw another. 
\end{itemize}

\subsection*{Permanent Cards}
Cards with a new type of effect, that guaranteed players permanent advantages, explained in \pagelink{Playerdecks}{Player Decks}.

\subsection*{Schools of Magic}
It's only thanks to many card effects from expansions that the \pagelink{Schools of Magic}{magic schools} suddenly gain more significance. Your heroes can specialize in certain magic schools to make their spells more powerful.

\subsection*{Unique Card effects with own components}
Several new cards from various expansions comes with own token components and bring some complex effects to the game. They all are explained in the separate section  \pagelink{Detailed Card Effects}{Detailed Card Effects}.

\subsection*{More of the same}
In addition to new gameplay mechanics, all expansions also expand the variety of existing components: new spells, new heroes, new artifacts, and much more...

\begin{expansion}{tower}
	\note{4}{
	The Tower expansion comes without any new gameplay mechanics, but offers a bit more of "More of the same". 
}
\end{expansion}
\pagebreak

\textbf{All contents on the following pages are added by specific expansions:}

\begin{expansion}{rampart,cove}
\subsection*{War Machines}
  A new card type with permanent effects, explained in \pagelink{War Machines}{Deck-building}.
\end{expansion}

\vspace*{1em}
\begin{expansion}{cove}
	\subsection*{Sea Map Tiles}
	In this expansion, your heroes can set sail and explore \pagelink{Map elements}{Sea Map Tiles}, whose bring a lot new sea locations to Antagarich. 
\end{expansion}

\vspace*{1em}
\begin{expansion}{stronghold}
	\subsection*{Subterranean Map Tiles}
	Exploring the subterranean world is your heroes new mission in this expansion. How to reach this subterranean world is explained in \pagelink{Subterranean Map Tiles}{Map elements}.
\end{expansion}

\vspace*{1em}
\begin{expansion}{stronghold}
	\subsection*{Spell Scroll Cards}
	Your heroes can find these mystical spell scrolls to cast additional spells, even if they are not so familiar with magic. How to handle this game mechanic is explained in \pagelink{War Machines}{Deck-building}.
\end{expansion}

\vspace*{1em}
\begin{expansion}{conflux}
	\subsection*{Elemental Map Tiles}
	Let your heroes travel through \pagelink{Elemental Map Tiles}{these Map Tiles} to cast their spells more powerfully.
\end{expansion}

\vspace*{1em}
\begin{expansion}{conflux}
	\subsection*{Summon Elemental}
	A unique type of spell cards to summon elemental units, which support you in your combat, explained in \pagelink{Detailed Card Effects}{Detailed Card Effects}.
\end{expansion}

\vspace*{1em}
\begin{expansion}{conflux}
	\subsection*{Monolith Map Locations}
	These new Map Locations allow teleporting from one to the other. 
	More information in \pagelink{Discover Location Tokens}{Map Elements} and \pagelink{Monolith}{All Map Locations}.
\end{expansion}

\vspace*{1em}
\begin{expansion}[before=\vspace*{-11mm}]{fortress}
  \subsection*{\pagetarget{Events}{Events}}
  Event Cards\index{Event Cards} may be used in games with more than one player.
  Shuffle the Event Deck during setup.
  At the start of each Resource Round (except the first Round), draw and read the next Event Card after receiving Resources.
  The first Event is drawn by the starting player.
  \textbf{Change the player who draws the Event in a clockwise order} every time a new Event is drawn.
  Resolve any effects in clockwise order starting with the player who drew the Card.
  Any Cards which were revealed as a part of resolving an Event should be shuffled back into their respective Decks afterwards.

  \medskip

  \begin{minipage}[h]{\linewidth}
    \vspace{0.1pt}
    \centering
    \begin{scriptsize}
      \begin{tikzpicture}
        \draw (0, 0) node[inner sep=0] {\makebox[\linewidth][c]{\includegraphics[width=\linewidth]{\cards/event.png}}};
        \draw (1, 1.8) node {\encircle{\phantom{.}1\phantom{.}}};
        \draw (-2, 0.4) node {\encircle{\phantom{.}2\phantom{.}}};
        \draw (2.5, -1.1) node {\encircle{\phantom{.}3\phantom{.}}};
      \end{tikzpicture}
    \end{scriptsize}
    \footnotesize
    \textbf{\textit{\textcolor{darkcandyapplered}{Event Card}}}
    \begin{multicols}{2}
      \begin{itemize}
        \item[\textbf{1.}] Name
        \item[\textbf{2.}] Fluff
        \item[\textbf{3.}] Effect
        \item[\textbf{\phantom{.}}] \phantom{.}
      \end{itemize}
    \end{multicols}
  \end{minipage}
\end{expansion}

\vspace*{1em}
\begin{expansion}{inferno}
	\subsection*{Summoning}
	A unique unit ability, explained in the \pagelink{Summoning}{Units} section.
\end{expansion}

\vspace*{1em}
\begin{expansion}{inferno}
	\subsection*{Empowered Statistic Cards}
	A new card type which improve your regular Statistic Cards, explained in \pagelink{Empowered Statistic}{Deck-building}.
\end{expansion}

\vspace*{1em}
\begin{expansion}{inferno}
	\subsection*{Random Town}\index{Random Town}
	A new \pagelink{Random Town}{Map Locations} with unique rules, that it makes it worth to mention.
\end{expansion}
\end{multicols*}

\begin{expansion}{battlefield}
  \begin{multicols*}{2}
  \subsection*{Battlefield Board in regular games}
  We recommend using the Battlefield Combat board only for \textbf{player vs. player Combats}, but it is also possible to use it in Neutral Combats.
  Using the Battlefield Board requires using \pagelink{Miniatures}{Miniatures} instead of cards.
  
  \medskip	
  \subsection*{Battlefield Board in Neutral Combats}
  If you decide to use the Battlefield Board in Combats with Neutral Units, ignore the round limit: 
  At the end of each combat round you may freely decide to retreat or to gain an additional Combat round with no need to spend a MP \svg{movement}. 
  Be aware, however, that this may significantly extend the playtime.
	
  \medskip
  \subsection*{Obstacle Tokens}
  On the Battlefield Board you place obstacle Tokens, which are divided into three types:
  Effect, Obstacle and Wall/Gate. 
  Unless otherwise stated these tokens count and work as \pagelink{Combatterminology}{Combat Obstacles}.
  To place Effect Obstacles on the battlefield Board use the tokens and put the Card beside the battlefield as a reminder. 
  Effect Obstacles may be entered by units, if the card text allows it. 
  \vspace*{1em}
  \begin{center}
  \includegraphics[width=0.8\linewidth]{\images/test_obstacle_token.png}
  \end{center}
  \vspace*{1em}
  \note{3}{
  	All Unit Miniatures also count as obstacles. 
  }
  \columnbreak
  
  \subsection*{Rules Changes in Combat Setup}
  \begin{itemize}
  	\item \textbf{Place obstacles} – Starting with the attacking player, both players take turns chosing and placing one obstacle token on the battlefield board until \textbf{all obstacle tokens are placed}. No obstacle can be adjacent to another Obstacle or to any player's deployment zone.
  \end{itemize}
  	\includegraphics[width=0.9\linewidth]{\examples/test-battlefield.png}
  \begin{itemize}
  	\item \textbf{Place units}  – Starting from the attacking player, the players choose a deployment zone and take turnd placing their units \textbf{one by one}. When the last unit is placed, the Combat begins. Only on player vs. player combats the players ca decide to allow placing \textbf{up to 7 units} (instead of 5).
  \end{itemize}
  \end{multicols*}
\end{expansion}

\newpage
\begin{expansion}{battlefield}
  \begin{multicols*}{2}
  \begin{itemize}
  	\item \textbf{Siege with Walls and Gate} – During a Siege against a town with citadel only Walls and Gate Obstacles are placed. Place no other obstacles. 
  \end{itemize}
  \includegraphics[width=1.05\linewidth]{\examples/test_battlefield_siege.png}
  \subsection*{Rules Changes during Combat}
  \begin{itemize}
  	\item \textbf{Unit movement} – Each Unit's movement is equal to its initiative value: a Unit of initiative 8 can move up to 8 spaces.
  	\item \textbf{Ranged movement}  – Ranged \svg{unit_ranged} Units can either move or attack, not both.
    \columnbreak
  	\item \textbf{Combat Penalty} – Ranged \svg{unit_ranged} Units suffer a combat penalty when attacking an adjacent unit or a unit that is \textbf{8 or more} spaces away from them.
  	\item \textbf{Initiative token} – At the start of combat, the attacking player gains the Initiative token. 
  	Use it to break any initiative ties of opposing units: The player with the token activates first \textbf{one of his units of the same initiative value}.
  	Then, the players alternate in activating all of their Units of that same initiative value one by one.
  	Once all units with that initiative value are activated, pass the Initiative Token to the other player and continue activating the units with the next lowest value in the same way.
  	\item During a siege, defending units may enter the Gate Token \textbf{only at the two middle hexes}, where the gate is printed on. Attacking the Gate Token is possible on all four hexes.
  \end{itemize}
  \includegraphics[width=1.05\linewidth]{\examples/test_gate_token.png}
  % Probably here is a asset sensful similar to naval battle mal with green arrows for allowed moves and red arrows for frbidden moves. 
  \end{multicols*}
\end{expansion}

\begin{multicols*}{2}
\vspace*{1em}
\begin{expansion}{battlefield}
	\subsection*{Morale Cards}
	An alternative Morale gameplay system, which you can use instead of the tokens, explained in \pagelink{Morale}{Player Turns}.
\end{expansion}
\end{multicols*}

\begin{expansion}[before=\vspace*{-11mm}]{navalbattles}
  \begin{multicols*}{2}
  \subsection*{\pagetarget{Creature Banks Rules}{Creature Banks}}
  Playing with Creature Banks is an \textbf{optional rule}.
  If you decide to play with Creature Banks, the following rules apply.

  \bigskip
  \subsection*{Setup}
  Separate the Creature Bank Tokens into two piles (for Near and Far Map Tiles) based on the numerals on their backs.
  Shuffle each pile and place them near the Map.
  Place the Creature Bank Unit Cards face up near the Neutral Unit Decks.
  There is no need to shuffle these Cards.
  Shuffle all Stack Tokens and place them near the Creature Bank Unit Cards.

  \bigskip
  \begin{multicols*}{2}
    \begin{center}
      \includegraphics[width=1.0\linewidth]{\images/creature_bank_near.png}\\
      \vspace{-0.5em}\textbf{\footnotesize\textit{\textcolor{darkcandyapplered}{Creature Bank (Near)}}}\\
      \vspace*{\fill}
      \includegraphics[width=0.5\linewidth]{\images/stack_token.png}\\
      \textbf{\footnotesize\textit{\textcolor{darkcandyapplered}{Stack Token}}}\\
      \columnbreak
      \includegraphics[width=\linewidth]{\cards/creature_bank_unit.png}\\
      \textbf{\footnotesize\textit{\textcolor{darkcandyapplered}{Creature Bank Unit Card}}}
    \end{center}
  \end{multicols*}

  \subsection*{Placing Creature Bank Tokens}
  When you discover a Near or Far Map Tile, you may choose to draw a Creature Bank Token of the corresponding type, and place it on the Tile's Blocked Field.\par

  \begin{center}
    \transparent{0.2}\includegraphics[width=1.1\linewidth]{\art/mermaid.png}
  \end{center}
  \columnbreak

  \subsection*{Visiting a Creature Bank\\Location}
  When you enter a Creature Bank Location, you must defeat the corresponding Creature Bank Units defending it.
  If you win, resolve the Field's effect and mark it with a Black Cube.
  Because these Fields don't have any Difficulty Level, Quick Combat never happens, there is no Round limit, no need to spend MP to extend Combat, and you don't gain any \svg{experience}.
  The defending Units and all effects of all Creature Bank Locations are listed in \pagelink{Creature Bank List}{All Map Locations}.

  \subsection*{Combat in a Creature Bank}
  When you start Combat at a Creature Bank, place up to 5 of your Units in the player deployment zone.
  Depending on the Creature Bank you are visiting, take the corresponding Units from the Creature Bank Unit Cards and place them randomly in the Neutral Unit deployment zone.

  \bigskip
  \begin{center}
    \begin{tikzpicture}
      \draw (0, 0) node[inner sep=0] {\makebox[\linewidth][c]{\includegraphics[width=\linewidth]{\images/creature_bank_combat_board.png}}};
      \draw (-1.5, 0.6) node {\includegraphics[width=0.15\linewidth]{\images/P.png}};
      \draw (-1.5, -0.5) node {\includegraphics[width=0.15\linewidth]{\images/P.png}};
      \draw (0, 0.6) node {\includegraphics[width=0.15\linewidth]{\images/P.png}};
      \draw (0, -0.5) node {\includegraphics[width=0.15\linewidth]{\images/P.png}};
      \draw (1.5, 0.6) node {\includegraphics[width=0.15\linewidth]{\images/P.png}};
      \draw (1.5, -0.5) node {\includegraphics[width=0.15\linewidth]{\images/P.png}};
      \draw (2.9, 1.7) node {\includegraphics[width=0.15\linewidth]{\images/N.png}};
      \draw (2.9, -1.6) node {\includegraphics[width=0.15\linewidth]{\images/N.png}};
      \draw (-3.1, 1.7) node {\includegraphics[width=0.15\linewidth]{\images/N.png}};
      \draw (-3.1, -1.6) node {\includegraphics[width=0.15\linewidth]{\images/N.png}};
    \end{tikzpicture}
  \end{center}
  \textbf{\footnotesize\textit{\textcolor{darkcandyapplered}{P -- player deployment zone}}}\\
  \textbf{\footnotesize\textit{\textcolor{darkcandyapplered}{N -- Neutral Units deployment zone}}}\\
  \bigskip

  Based on the Scenario's Difficulty Level, take a number of \pagelink{Stack Tokens}{Stack Tokens} and place them randomly on up to four different Creature Bank Unit Cards:
  \begin{itemize}
    \item \textbf{Easy}: 1 token
    \item \textbf{Normal}: 2 tokens
    \item \textbf{Hard}: 3 tokens
    \item \textbf{Impossible}: 4 tokens
  \end{itemize}

  \end{multicols*}
\end{expansion}

\begin{multicols*}{2}
\begin{expansion}[before=\vspace*{-11mm}]{navalbattles}
  After Combat, return all Creature Bank Units back to the Creature Bank Units Cards pile.
  If you win, you claim the Creature Bank's reward, as well as an additional reward for the Stacked Units you defeated.
  Some rewards allow you to gain a Stacked Creature Bank Unit Card.
  When you gain it, draw a random Stack Token and place it on that Unit.

  \subsection*{\pagetarget{Stack Tokens}{Stack Tokens}}
  \setlength\intextsep{0pt}
  \setlength\columnsep{1em}
  \begin{wrapfigure}{r}{0.2\linewidth}
    \includegraphics[width=\linewidth]{\images/stack_token.png}
  \end{wrapfigure}
  Each Stack Token modifies a Unit's statistics in the following way: +1~\svg{attack}, +1~\svg{defense}, +1~\svg{health_points}, or +2~\svg{initiative}.
  Furthermore, each Unit with a Stack Token is now a Stacked Unit, which is similar to reinforced Faction Units: instead of removing the Unit Card from the Combat Board when it takes damage equal to or greater than its HP, discard the Stack Token from that Unit and deal any leftover damage (if any), deducting it from the new HP.
  Creature Bank Units without a Stack Token are defeated normally.
\end{expansion}

\bigskip

\begin{expansion}{navalbattles}
  \subsection*{Naval Combat Board}
  Using the \pagelink{combat}{Naval battle board} lets your heroes fight a combat on ships, what is more challenging than combats on the normal combat board.
\end{expansion}

\vspace*{1em}
\begin{expansion}{navalbattles}
  \subsection*{Empowered Ability Cards}
  This new type of cards improves your heroes ability cards and it is explained in \pagelink{Deck-buildung}{Deck-building}.
\end{expansion}

\begin{expansion}{stretchgoals2}
  \subsection*{\pagetarget{Pandora Card}{Pandora's Box Cards}}
  During setup, shuffle the Pandora's Box Cards.
  When you Visit a \pagelink{Pandora Box}{Pandora's Box}, you may optionally draw a Pandora's Box Card and resolve it instead of the location's regular effect.
  You still have to mark the Field with a Black Cube.

  \medskip
  \begin{center}
    \includegraphics[width=0.6\linewidth]{\cards/pandora.png}
  \end{center}
\end{expansion}

\vspace*{1em}
\subsection*{You haven't had enough yet?}
Take a look at the fanmade expansions or contents that the \textbf{HoMM3 BG Community} has developed:
  \begin{itemize}
    \item The \href{https://github.com/qwrtln/Homm3BG-mission-book}{Fanmade Mission Book} contains many well playtested scenarios with unique scenario rules for even more spice in your games. (ADD LINK AND QR CODE)
	\item The \href{https://github.com/piotrbruzda/Homm3BG-FactoryRulebook}{fanmade factory expansion} introduces the beloved missing faction. Print it yourself and learn a new faction to master. (LINK AND QR CODE)
  \end{itemize}



\columnbreak

\includegraphics[width=\linewidth]{\art/oceanid.jpg}

\end{multicols*}
