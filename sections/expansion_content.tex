% !TeX spellcheck = en_US
\addsection{Expansion Content}{\skills/pathfinding.png}

\begin{multicols*}{2}

Numerous expansions have been released for Heroes III: The Board Game, introducing new gameplay elements.
Most expansion rules are integrated directly into the relevant sections of this rulebook where they belong.
Below you find an overview of all expansion content, either as general rules added by multiple expansions, or by unique rules from specific expansions.

\subsection*{New Map Locations}
Each expansion adds new map locations to the game, which your Heroes can explore.
They are explained in \pagelink{All Map Locations}{All Map Locations}.

\subsection*{Permanent Cards}
Added by multiple expansions, explained in \pagelink{Playerdecks}{Player Decks}.

\vspace*{1em}
\begin{expansion}{rampart,cove}
\subsection*{War Machines}
  Explained in \pagelink{War Machines}{Deck-building}.
\end{expansion}

\vspace*{1em}
\begin{expansion}{inferno}
\subsection*{Summoning}
  Explained in the \pagelink{Summoning}{Units} section.
\end{expansion}

\vspace*{1em}
\begin{expansion}{inferno}
\subsection*{Empowered Statistic Cards}
  Explained with regular Statistic Cards in \pagelink{Empowered Statistic}{Deck-building}.
\end{expansion}

\vspace*{1em}
\begin{expansion}{inferno}
  \subsection*{Random Town}\index{Random Town}
  See it in \pagelink{Random Town}{All Map Locations}.
\end{expansion}

\begin{expansion}{fortress}
  \subsection*{\pagetarget{Events}{Events}}
  Event cards\index{Event Cards} may be used in games with more than one player.
  Shuffle the Event Deck during setup.
  At the start of each Resource Round (except the first Round), draw and read the next Event Card after receiving Resources.
  The first Event is drawn by the starting player.
  \textbf{Change the player who draws the Event in a clockwise order} every time a new Event is drawn.
  Resolve any effects in clockwise order starting with the player who drew the Card.
  Any cards which were revealed as a part of resolving an Event should be shuffled back into their respective Decks afterwards.

  \medskip

  \begin{minipage}[h]{\linewidth}
    \vspace{0.1pt}
    \centering
    \begin{scriptsize}
      \begin{tikzpicture}
        \draw (0, 0) node[inner sep=0] {\makebox[\linewidth][c]{\includegraphics[width=\linewidth]{\cards/event.png}}};
        \draw (1, 1.8) node {\encircle{\phantom{.}1\phantom{.}}};
        \draw (-2, 0.4) node {\encircle{\phantom{.}2\phantom{.}}};
        \draw (2.5, -1.1) node {\encircle{\phantom{.}3\phantom{.}}};
      \end{tikzpicture}
    \end{scriptsize}
    \footnotesize
    \textbf{\textit{\textcolor{darkcandyapplered}{Event Card}}}
    \begin{multicols}{2}
      \begin{itemize}
        \item[\textbf{1.}] Name
        \item[\textbf{2.}] Fluff
        \item[\textbf{3.}] Effect
        \item[\textbf{\phantom{.}}] \phantom{.}
      \end{itemize}
    \end{multicols}
  \end{minipage}
\end{expansion}

\vspace*{\fill}
\end{multicols*}
