% !TeX spellcheck = en_US
\addsection{Expansion Content}{\skills/pathfinding.png}

\begin{multicols*}{2}

Numerous expansions have been released for Heroes III: The Board Game, introducing new gameplay elements.
Most expansion rules are integrated directly into the relevant sections of this rulebook where they belong.
Below you'll find an overview of all expansion content, either as general rules added by multiple expansions, or by unique rules from specific expansions.

\subsection*{New Map Locations}
Each expansion adds new map locations to the game, which your Heroes can explore.
They are explained in \pagelink{All Map Locations}{All Map Locations}.

\subsection*{Permanent Cards}
Added by multiple expansions, explained in \pagelink{Playerdecks}{Player Decks}.

\vspace*{1em}
\begin{expansion}{rampart,cove}
\subsection*{War Machines}
  Explained in \pagelink{War Machines}{Deck-building}.
\end{expansion}

\vspace*{1em}
\begin{expansion}{inferno}
\subsection*{Summoning}
  Explained in the \pagelink{Summoning}{Units} section.
\end{expansion}

\vspace*{1em}
\begin{expansion}{inferno}
\subsection*{Empowered Statistic Cards}
  Explained with regular Statistic Cards in \pagelink{Empowered Statistic}{Deck-building}.
\end{expansion}

\vspace*{1em}
\begin{expansion}{inferno}
  \subsection*{Random Town}\index{Random Town}
  See it in \pagelink{Random Town}{All Map Locations}.
\end{expansion}

\begin{expansion}{fortress}
  \subsection*{\pagetarget{Events}{Events}}
  Event cards\index{Event Cards} may be used in games with more than one player.
  Shuffle the Event Deck during setup.
  At the start of each Resource Round (except the first Round), draw and read the next Event Card after receiving Resources.
  The first Event is drawn by the starting player.
  \textbf{Change the player who draws the Event in a clockwise order} every time a new Event is drawn.
  Resolve any effects in clockwise order starting with the player who drew the Card.
  Any cards which were revealed as a part of resolving an Event should be shuffled back into their respective Decks afterwards.

  \medskip

  \begin{minipage}[h]{\linewidth}
    \vspace{0.1pt}
    \centering
    \begin{scriptsize}
      \begin{tikzpicture}
        \draw (0, 0) node[inner sep=0] {\makebox[\linewidth][c]{\includegraphics[width=\linewidth]{\cards/event.png}}};
        \draw (1, 1.8) node {\encircle{\phantom{.}1\phantom{.}}};
        \draw (-2, 0.4) node {\encircle{\phantom{.}2\phantom{.}}};
        \draw (2.5, -1.1) node {\encircle{\phantom{.}3\phantom{.}}};
      \end{tikzpicture}
    \end{scriptsize}
    \footnotesize
    \textbf{\textit{\textcolor{darkcandyapplered}{Event Card}}}
    \begin{multicols}{2}
      \begin{itemize}
        \item[\textbf{1.}] Name
        \item[\textbf{2.}] Fluff
        \item[\textbf{3.}] Effect
        \item[\textbf{\phantom{.}}] \phantom{.}
      \end{itemize}
    \end{multicols}
  \end{minipage}
\end{expansion}

\end{multicols*}


\begin{expansion}[before=\vspace*{-11mm}]{navalbattles}
  \begin{multicols*}{2}
  \subsection*{\pagetarget{Creature Banks Rules}{Creature Banks}}
  Playing with Creature Banks is an \textbf{optional rule}.
  If you decide to play with Creature Banks, the following rules apply.

  \bigskip
  \subsection*{Setup}
  Separate the Creature Bank tokens into two piles (for Near and Far Map Tiles) based on the numerals on their backs.
  Shuffle each pile and place them near the map.
  Place the Creature Bank Unit Cards face up near the Neutral Unit decks.
  There is no need to shuffle this deck.
  Shuffle all Stack Tokens and place them near the Creature Bank Unit Cards.

  \bigskip
  \begin{multicols*}{2}
    \begin{center}
      \vspace*{\fill}
      \includegraphics[width=0.5\linewidth]{\images/stack_token.png}\\
      \textbf{\footnotesize\textit{\textcolor{darkcandyapplered}{Stack Token}}}\\
      \vspace*{\fill}
      \columnbreak
      \includegraphics[width=\linewidth]{\cards/creature_bank_unit.png}\\
      \textbf{\footnotesize\textit{\textcolor{darkcandyapplered}{Creature Bank Unit Card}}}
    \end{center}
  \end{multicols*}

  \subsection*{Placing Creature Bank Tokens}
  When you discover a Near or Far Map Tile, you may choose to replace one of its blocked fields with a Creature Bank token of the corresponding type.\par

  \begin{center}
    \transparent{0.2}\includegraphics[width=1.1\linewidth]{\art/mermaid.png}
  \end{center}
  \columnbreak

  \subsection*{Visiting a Creature Bank\\Location}
  When you enter a Creature Bank Location, you must defeat the corresponding Creature Bank Units defending it.
  If you win, resolve the field's effect and mark it with a black cube.
  Because these fields don't have any difficulty level, Quick Combat never happens, there is no round limit, no need to spend MP to extend the combat, and you don't gain any \svg{experience}.
  The defending Units and all effects of all Creature Bank Locations are listed in \pagelink{Creature Bank List}{All Map Locations}.

  \subsection*{Combat in a Creature Bank}
  When you start a Combat at a Creature Bank, place up to 5 of your Units in the player deployment zone.
  Depending on the Creature Bank you are visiting, take the corresponding Units from the Creature Bank Unit Cards and place them randomly in the Neutral Unit deployment zone.

  \bigskip
  \begin{center}
    \includegraphics[width=\linewidth]{\images/creature_bank_combat_board.png}\\
    \textbf{\footnotesize\textit{\textcolor{darkcandyapplered}{Green -- player deployment zone}}}\\
    \textbf{\footnotesize\textit{\textcolor{darkcandyapplered}{Red -- Neutral Units deployment zone}}}\\
  \end{center}
  \bigskip

  Based on the scenario's difficulty level, take a number of Stack tokens and place them randomly on up to four different Creature Bank Unit Cards:
  \begin{itemize}
    \item \textbf{Easy}: 1 token
    \item \textbf{Normal}: 2 tokens
    \item \textbf{Hard}: 3 tokens
    \item \textbf{Impossible}: 4 tokens
  \end{itemize}

  \end{multicols*}
\end{expansion}

\begin{multicols*}{2}
\begin{expansion}[before=\vspace*{-11mm}]{navalbattles}
  Each Stack token modifies a Unit's statistics the following way: +1 \svg{attack}, +1 \svg{defense}, +1 \svg{health_points} or +2 \svg{initiative}.
  Furthermore, each Unit with a Stack token is now a Stacked Unit which is similar to reinforced Faction Units: Instead of removing the Unit Card from the combat board when it takes damage equal to or greater than its HP, discard the Stack token from that Unit and deal any leftover damage (if any), deducting it from the new HP.
  Creature Bank Units without a Stack token are defeated normally.
  After the Combat, return all Creature Bank Units back to their Unit Cards.
  If you won, you will claim the Creature Bank's reward, as well as an additional reward for every Stacked Unit you defeat.
  Some rewards allow you to gain a Stacked Creature Bank Unit Card.
  When you gain it, draw a random Stack token and place it on that Unit.
\end{expansion}
\columnbreak

\includegraphics[width=\linewidth]{\art/oceanid.jpg}

\end{multicols*}
