% !TeX spellcheck = en_US
\pagetarget{AI Units}{\addsection{Combat against AI}{\spells/fortune.png}}

\begin{multicols*}{2}
These rules apply during Combat in \textbf{Solo}\index{Solo Mode} and \textbf{Cooperative} Scenarios.
The rules for AI unit placement during setup are described in \pagelink{CombatAISetup}{Combat section}.

When Neutral enemies or AI Heroes activate a unit, they follow a set of automatic instructions:\par

\begin{itemize}
  \item If it's Solo Scenario, draw and resolve the next \pagelink{AI Deck}{AI card}.

  \item Enemy Ground \svg{unit_ground} and Flying \svg{unit_flying} units prioritize attacking units of the \textbf{same} tier.
  If this is impossible, they attack the unit of a lower tier (in tier \textbf{descending} order, down to bronze), and if that is also impossible, they attack the unit of a higher tier (in tier \textbf{ascending} order).\par

  \textit{Example: \svg{golden}\svg{unit_ground} has this priority:
  \svg{golden}
  - \svg{silver}
  - \svg{bronze}
  - \svg{azure}.}

  \item Ranged \svg{unit_ranged} units prioritize attacking other Ranged \svg{unit_ranged} units of the same tier, then lower tier, and finally higher tier, using the same tier order as above.
  If there are no Ranged \svg{unit_ranged} units for them to target, they prioritize Ground \svg{unit_ground} and Flying \svg{unit_flying} units in the same tier order.\par

  \textit{Example: \svg{silver}\svg{unit_ranged} has this priority:
  \svg{silver}\svg{unit_ranged}
  - \svg{bronze}\svg{unit_ranged}
  - \svg{golden}\svg{unit_ranged}
  - \svg{azure}\svg{unit_ranged}
  - \svg{silver}\svg{unit_ground}\svg{unit_flying}
  - \svg{bronze}\svg{unit_ground}\svg{unit_flying}
  - \svg{golden}\svg{unit_ground}\svg{unit_flying}
  - \svg{azure}\svg{unit_ground}\svg{unit_flying}.}
\end{itemize}

In both cases, if there's more than one valid target, they attack the closest one.
If there's ever a tie between equally valid targets, the player chooses which unit is attacked.\par

Enemy units cannot \pagelink{Defend}{defend} unless instructed to.

\subsection*{AI Siege}
When the Walls and Gate are mentioned in Combat preparation but no additional information on how to arrange them is given, arrange them just like a human player would:
place the Gate in front of the \svg{unit_ground} unit with the highest \svg{initiative}.
By default, the units do not attack the Walls -- they would rather fly over them to attack their target or move towards it through the Gate.
If it is not possible, they take a \pagelink{Defend}{Defend} Action.

The \pagelink{Walls}{Arrow Tower} is treated like a Silver \svg{silver}\svg{unit_ranged} unit.
However, if there are multiple equally valid targets, it will attack the target closest to perishing (has the smallest difference between \svg{health_points} and current \svg{damage}).

\subheader{Casting complex spells}
In some campaigns enemies use a number of spells whose effects are not fully compatible with the standard use of AI Magic \svg{magic} cards.
To fully use their effects, here are extended descriptions of how AI Heroes should use each of these spells.
Of course, the scenario can override this.

\subsubsection*{Spells attacking multiple targets (like \wikilink{spells/fireball}{Fireball} and \wikilink{spells/chain_lightning}{Chain Lightning})}
When activated, target any unit with one or two adjacent units from the player’s army, prioritizing the groups where there are more higher-tier units.
If there is more than one valid group, attack the one that is between the closest to perishing (has the smallest difference between \svg{health_points} and current \svg{damage}).
If there is still more than one valid target, then you can choose which unit is attacked.
If there are no player units adjacent to one another, target units that are not adjacent to any of the AI units.
If that is also not possible, do not use this spell -- instead, skip the AI \svg{magic} card that activated this effect and put it on the bottom of the Enemy AI deck, then shuffle this spell back to the Enemy Spell deck.

\subsubsection*{Instant Defense spells (like \wikilink{spells/stone_skin}{Stone Skin)}}
When activated, put this card on the side of the Combat board, then put a Defense token on the unit with the highest \svg{defense} to represent the card’s effect -- it stays there until the defense is resolved.
If there already is a Defense token on that unit, choose another one in the order of decreasing \svg{defense}.
In case of a tie in \svg{defense} value, give preference to the unit of the highest tier and then to the greatest value of \svg{damage}.

\subsubsection*{Healing spells (like \wikilink{spells/cure}{Cure)}}
When activated, remove the \svg{damage} from the AI unit with the greatest value of \svg{damage} tokens, starting with the highest tier available.
If no AI unit has any \svg{damage}, put the AI \svg{magic} card that triggered the spell at the bottom of the Enemy AI deck, then shuffle this Spell card back to the Enemy Spell Deck.

\subsubsection*{Single-round buffs (like \wikilink{spells/fire_shield}{Fire Shield})}
% TODO FIFO or LIFO? do they all activate at the same time during first unit next round?
When activated, check the tier of the unit on which you are about to cast the spell and count how many units of the same or higher tier there are on the board.
If more than half of them have already activated this turn, do not cast the spell now -- instead, place it on the side of the Combat board and play it when the first AI unit activates in the next combat round.
Skip drawing the AI card for that activation.

\subsubsection*{Attack-weakening spells (like \wikilink{spells/weakness}{Weakness})}
When activated, if the AI’s activated unit is about to perform an attack that will provoke a Retaliation, cast this spell on the Retaliating enemy to lower their \svg{attack}.
If the AI’s unit causes no Retaliation, do not cast this spell -- instead, ignore the AI card that activated the spell and put it at the bottom of the Enemy AI deck, then shuffle the Spell card back to the Enemy Spell deck.

{\begin{center}
    \includegraphics[width=\linewidth,height=\textheight,keepaspectratio]{\art/titan.jpg}
\end{center}}

\end{multicols*}
