% !TeX spellcheck = en_US
\addsection{Heroes}{\skills/sorcery.png}

\begin{multicols*}{2}

\pagetarget{Heroes}{Players} always control a Main Hero and may additionally also recruit a Secondary Hero.
A ``player's Hero'' may refer to either of them.
Heroes are used to perform Movement Actions on the game Board and to start Combats against enemies in order to reach a Scenario victory condition.

\subsection*{Main Hero}
The Main Hero\index{Main Hero} is represented by its chosen model, Hero Board, and your Deck.
Each Faction's Main Hero has 3 \svg{movement}.
Only the Main Hero can use your Deck.\par
Each Main Hero starts the game at Level 1 and can advance up to Level 7 by gaining Experience.
Experience is gained from \pagelink{Combatexperience}{winning Combat}, Visiting certain \pagelink{All Map Locations}{Locations} and the \pagelink{Treasure Die}{Treasure Die \svg{treasure}}.
Gaining 1 Experience\index{Experience} is represented by the symbol \svg{experience}.

\subsection*{\pagetarget{Secondary}{Secondary Hero}}
If you control a Town or a Settlement, a Secondary Hero\index{Secondary Hero} can be Hired by flipping your \textbf{Population Token} and paying 10 \svg{gold}.\par
\note{5}{Units \textbf{cannot} be \pagelink{Units}{Recruited or Reinforced} while using the Population Token to recruit a Secondary Hero.}\par
Your Secondary Hero uses the remaining Hero model of your Faction.
You may wish to mark this model with a token such as a Faction Cube to differentiate it from the Main Hero.
After Hiring a Secondary Hero, place the model in a Town or Settlement you control.
\textbf{You can only have one Secondary Hero at a time}.\par
\columnbreak
Secondary Heroes have \textbf{2 \svg{movement}}; when you gain a Secondary Hero, take an additional set of 2 Movement Tokens to represent their MP.
They do not have their own Hero Board, \textbf{cannot gain Experience}, \textbf{cannot play cards from your Deck during Combat}, but use \textbf{the same units} your Main Hero does.
If a Secondary Hero gains any cards, place them into your hand as normal (see \pagelink{Playerdecks}{Deck-building}).
Secondary Heroes are considered to have the same Level as the Main Hero for the purposes of resolving \pagelink{Quick}{Quick Combat}.\par
If your Secondary Hero is attacked by an enemy Hero, you can choose to have that Hero be \pagelink{Endcombat}{instantly defeated instead of fighting a Combat}.
When a Secondary Hero is defeated, remove them from the game.
They can be Recruited again with another use of the Population Token.\par

\vspace*{\fill}
\hspace{2em}
{\transparent{0.2}\includegraphics[width=\linewidth]{\art/clone.png}}
\vspace*{\fill}

\end{multicols*}

\clearpage

\pagetarget{Herocard}{\subheader{Hero Board Anatomy}}\index{Hero Board}
\bigbreak
\begin{figure}[h]
  \begin{minipage}[t]{0.5\textwidth}
    \vspace{0pt}
    \begin{enumerate}[itemsep=5pt]
      \item \textbf{Name} – The Hero's name.
        Used for identification.
        Has no gameplay effect.
      \item \textbf{Class} – The Hero's class.
        Has no gameplay effect.
      \item \textbf{Type} – The Hero's type (Might \svg{might} or Magic \svg{magic}).
        Determines the amount of Magic Arrow Spells in your Starting Deck (1 or 2 respectively).
      \item \textbf{Faction Color} – Reminder for the color of the Faction's Cubes and miniatures.
      \item \textbf{Attack} – Number of Attack cards in your Starting Deck.
      \item \textbf{Defense} – Number of Defense cards in your Starting Deck.
      \item \textbf{Power} – Number of Power cards in your Starting Deck.
      \item \textbf{Knowledge} – Number of Knowledge cards in your Starting Deck.
      \item \textbf{Starting Ability} – Reminder for the unique Ability card the Hero starts with.
      \item \textbf{Hero Specialty} – Reminder for the Specialty cards the Hero adds to their Deck at the start of the game and after specific Level ups.
        Each Hero has three Specialty cards.
      \item \textbf{Level Tracker} – Whenever a Main Hero gains 1 or more Experience \svg{experience}, move the Cube that number of steps on this track.
        When the Cube reaches the next slot on the upper row, the Hero gains a Level.
    \end{enumerate}
  \end{minipage}\hfill
  \begin{minipage}[t]{0.48\textwidth}
    \centering
    \vspace{0pt}
    \begin{scriptsize}
      \hspace*{2em}
      \begin{tikzpicture}
        \draw (0, 0) node[inner sep=0] {\makebox[\textwidth][c]{\includegraphics[width=\linewidth]{\cards/hero.png}}};
        \draw (2.2, 2.5) node {\encircle{\phantom{.}1\phantom{.}}};
        \draw (0.8, 1.9) node {\encircle{\phantom{.}2\phantom{.}}};
        \draw (3.5, 2.5) node {\encircle{\phantom{.}3\phantom{.}}};
        \draw (-0.1, 2.5) node {\encircle{\phantom{.}4\phantom{.}}};
        \draw (0, 1.25) node {\encircle{\phantom{.}5\phantom{.}}};
        \draw (1.1, 1.25) node {\encircle{\phantom{.}6\phantom{.}}};
        \draw (2, 1.25) node {\encircle{\phantom{.}7\phantom{.}}};
        \draw (3.25, 1.25) node {\encircle{\phantom{.}8\phantom{.}}};
        \draw (1, -0.2) node {\encircle{\phantom{.}9\phantom{.}}};
        \draw (3, -0.2) node {\encircle{10}};
        \draw (-1.7, -1.4) node {\encircle{11}};
      \end{tikzpicture}
    \end{scriptsize}
    \break
    \footnotesize{\textbf{\textit{\textcolor{darkcandyapplered}{Hero Board}}}}
    \scriptsize
    \begin{multicols}{2}
      \begin{itemize}
        \item[\textbf{1.}] \textbf{Name}
        \item[\textbf{2.}] \textbf{Class}
        \item[\textbf{3.}] \textbf{Type}
        \item[\textbf{4.}] \textbf{Faction Color}
        \item[\textbf{5.}] \textbf{Attack}
        \item[\textbf{6.}] \textbf{Defense}
        \item[\textbf{7.}] \textbf{Power}
        \item[\textbf{8.}] \textbf{Knowledge}
        \item[\textbf{9.}] \textbf{Starting Ability}
        \item[\textbf{10.}] \textbf{Specialty}
        \item[\textbf{11.}] \textbf{Level Tracker}
        \item[\textbf{\phantom{.}}] \phantom{.}
      \end{itemize}
    \end{multicols}
  \end{minipage}
\end{figure}

\begin{tikzpicture}[overlay]
  \node[opacity=0.2, rotate=20] at (12, -0.5) {\includegraphics[width=0.6\linewidth]{\art/griffin.png}};
\end{tikzpicture}

\clearpage

\pagetarget{Level}{\subheader{Level Effects}}
\begin{multicols}{2}
Main Heroes always start each Scenario at Level 1\index{Level} and may Level up by gaining Experience \svg{experience}.
The most common sources of gaining Experience are the \pagelink{Treasure Die}{Treasure Die \svg{treasure}} and \pagelink{Combatexperience}{Combat}.
Each new Level up requires \textbf{2 Experience}.
When a Main Hero reaches a new Level, resolve the effects of the Level up immediately.
Gaining Experience at Level 7 has no effect.\par
The Level Tracker on your Hero Board shows the following information:
\begin{itemize}
\item Your Main Hero's current Level and amount of Experience gained, shown by the Cube's position.
\item Your current Hand Limit \svg{hand}.
\item The number of \pagelink{Ability}{Expert Effects} \svg{expert} you may use during a Round.
\item At which Levels your Main Hero must \pagelink{Playerdecks}{Search} for a new \pagelink{Ability}{Ability card} or gain a \pagelink{Specialty}{Specialty card}.
Level numbers written in gold on the Level Tracker (\svg{level1}, \svg{level4} and \svg{level6}) give you a Specialty card, while silver Levels (\textbf{II}, \textbf{III}, \textbf{V}, \textbf{VII}) give you an Ability card.
\end{itemize}
\vfill\null
\columnbreak
List of all effects:
\begin{itemize}
\item \textbf{Level 1} – Your Hand Limit\index{Hand Limit} is 4.
Add your first Specialty card to your Deck.
\item \textbf{Level 2} – Search (2) the Ability Deck.
You may play 1 card for its Expert Effect per Round.
\item \textbf{Level 3} – Your Hand Limit is 5.
Search (2) the Ability Deck.
\item \textbf{Level 4} – Gain your second Specialty card.
You may play 2 cards for their Expert Effect per Round.
\item \textbf{Level 5} – Your Hand Limit is 6.
Search (2) the Ability Deck.
\item \textbf{Level 6} – Gain your third Specialty card.
You may play 3 cards for their Expert Effect per Round.
\item \textbf{Level 7} – Your Hand Limit is 7.
Search (2) the Ability Deck.
\end{itemize}

\end{multicols}

\begin{tikzpicture}[overlay]
  \node[opacity=0.2] at (9, -3) {\includegraphics[width=0.6\linewidth]{\art/cavalryman.png}};
\end{tikzpicture}
