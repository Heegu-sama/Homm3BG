\addsection[2]{Adventure and \break{}Skirmish Modes}{\spells/frenzy.png}\pagetarget{Adventure and Skirmish Modes}{}

\begin{expansion}{battlefield}
  \begin{multicols}{2}
  \subsection*{Overview}
  The Skirmish and Adventure Mode are similarly structured 1~vs.~1 game modes whose central idea is to engage in a tactical combat on the large \pagelink{Battlefield Combat}{Battlefield Board}.
  The two modes differ in the way players build their unit and Player Decks:
  \smallskip
  \begin{itemize}
    \item In \textbf{Skirmish Mode} the players simply get pre-constructed Decks and armies, on whose composition they have only a little influence.
      With their Decks the players jump directly into Combat.
    \item In \textbf{Adventure Mode}, a card-driven Adventure Phase is played before the final Combat, which replaces and simulates the discovery of the Map from regular games.
      In this phase, players use Adventure cards to gather Resources, expand their Town, and develop their Heroes, thus deciding for themselves which units to recruit and which types of cards to include in their Deck, rather than limited on pre-constructed sets.
  \end{itemize}

  \begin{center}
    {\transparent{0.2}{\includegraphics[width=\linewidth]{\art/quiet_eye_of_the_dragon.png}}}
  \end{center}\par

  \vspace*{\fill}
  \columnbreak

  \subsection*{Victory Condition}
  To achieve victory, the player must defeat all enemy units in the final Combat.

  \subsection*{Rule Ambiguity}
  \imagecaption{Important Note:}
  Both game modes were developed with only the \textbf{Core Game in mind}.
  Therefore, official rules leave many questions unanswered, especially (but not exclusively) concerning card effects from expansions.
  If you don't want to remove all Expansion cards from the game, you'll have no choice but to consult house rules.

  \smallskip
  The following note box provides some clues about missing clarifications from the developers.
  But when learning the rules you should skip this part for now.

  \smallskip
  \note{18}{Open Questions in Skirmish and/or Adventure Mode:
  \begin{itemize}
    \item Which else cards should be removed from the game (beside Diplomacy and Logistics)?
    \item Which cards should be removed in Skirmish (cause there are only Combat effects useful)?
    \item How to handle cards with Permanent effects in Adventure Mode ?
    \item How to handle cards with a "Remove" Effect (e.~g. Expert Learning) in Adventure Mode?
    \item How to handle cards with both a Combat and an Adventure Effect?
  \end{itemize}
  }\par

  \vspace*{\fill}
  \end{multicols}
\end{expansion}

\newpage
\begin{expansion}{battlefield}
  \subheader{Setup for both modes}
  \begin{multicols}{2}
    \subsection*{Common Setup Steps}
    \begin{enumerate}
      \item Place the Battlefield Board in the middle of the table.
      \item Each player chooses a Faction and a Hero of that Faction and takes its following components:
      \begin{itemize}
        \item Hero Board
        \item 3 Hero's Specialty cards and its starting Ability card
        \item 7 unit cards and their corresponding 7 unit miniatures
        \item 20 Faction Cubes
      \end{itemize}
      \item Prepare the Artifact, Spell and Ability Deck normally. Remove the following Ability cards from play: ``Diplomacy'', ``Logistics''.
      \item Prepare the Positive and Negative \pagelink{Morale Cards}{\textbf{Morale cards}} (including all cards with the battlefield symbol).
      \item Put following Tokens ready near the Battlefield Board: Black Cubes, Damage Tokens, Defense/Paralysis Tokens, Obstacles and the Initiative Token.
      \item Put the Attack and Resource Dice ready.
      \item Build your starting Deck as normal.
     \end{enumerate}

     \subsection*{Additional Steps for Adventure Mode}
     \begin{enumerate}[resume]
      \item Put also Resource Token and Treasure Dice ready.
      \item Each player takes the Town Board with following starting requirements:
      \begin{itemize}
        \item Income tracks at 10 \svg{gold}, 0 \svg{building_materials} and 0 \svg{valuablegreater}.
        \item Starting resources: 15 \svg{gold}, 3 \svg{building_materials} and 1 \svg{valuablegreater}.
        \item Start without Town Buildings and without units in your Unit Deck
      %https://boardgamegeek.com/thread/3284421/article/44125381#44125381 (clarification1)
      \end{itemize}
      \item Shuffle the Adventure card Deck and place it face down.
      \item Place the Round Tracker and place a Black Cube at Round 1.
      \item Set Hero Levels to 1.
  \end{enumerate}\par
  \medskip
  \subsection*{Start of game}
  After Setup, both players roll 2 \svg{resource_die}.
  The one with the highest results is the first player.
  Use the Initiative Token to track the current first player.
  Set your starting Deck aside for now.
  You only need it in the Final Combat.
  \end{multicols}
\begin{center}
  \includegraphics[width=0.9\linewidth]{\images/battlefield_setup.png}
\end{center}
\end{expansion}


\newpage
\begin{expansion}[before=\vspace*{-11mm}]{battlefield}
  \subheader{Adventure Mode}
  \begin{multicols}{2}
  \pagetarget{Adventure}{The Adventure Mode} is divided into two Phases: The \textbf{Adventure Phase} and the \textbf{Final Combat}. The latter works identically as in the Skirmish Mode.\par
  Before starting the game, both players decide for a game length by limiting the Adventure Phase's Round number:
  \begin{itemize}
    \item \textbf{Short} game: 9 Rounds.
    \item \textbf{Medium} game: 13 Rounds.
    \item \textbf{Large} game: 16 Rounds.
  \end{itemize}
  \subsection*{Adventure Phase}
  The Adventure Phase lasts a certain number of Rounds, depending on the chosen game length.
  Follow these steps in the following order during each Round.\par
    \begin{enumerate}
      \item \textbf{Move Round Tracker} – Skip this step in the first round.
      Depending on the type of Round, resolve following effects:
      \begin{itemize}
        \item In \textbf{Resource Rounds} both players gets income as normal.
        \item In \textbf{Astrologers Rounds} – instead of drawing Astrologers Proclaim cards - players increase one of their Income Tracks (\svg{gold},\svg{building_materials} or \svg{valuablegreater}) in one of the following ways:
        \begin{itemize}
          \item Choose your preferred Income Track by yourself.
          \item Determine the Income Track randomly by rolling a \svg{resource_die}. If you do, get the Basic income of that Resource immediately, which is 5 \svg{gold}, 2 \svg{building_materials} or 1 \svg{valuablegreater}.
        \end{itemize}
      \end{itemize}
      \item \textbf{Drawing Adventure cards} – Skip this step in the last round.
      The starting Player draws 3 Adventure cards, keeps one of them and passes the remaining 2 to the other player, who keeps one and discards the other.\par
      (\textbf{Note:} Only the starting player may resolve \svg{movement} effects to discard any number of drawn Adventure cards and redraw that many cards, hopefully to get a better selection.)\par
      \item \textbf{Player Turn} – Starting with the first player both players takes their turn by performing some of the following Actions in any order:
      \begin{itemize}
        \item You \textbf{resolve} the chosen Adventure card, or \textbf{discard} it, or \textbf{save it up} for later turns.
        If you save it place it hidden on your Hero Board.
        You can only save one Adventure card at a time.
        If you already have a saved Adventure card, you have to resolve or discard that card this turn.
        \item Resolve a previous saved Adventure card.
        You may only resolve \textbf{one Adventure card per turn}.
        \item Perform Town or Morale Actions or resolve your Hero's Level Ups the same way as in regular games.
        \item You can play effects of \pagelink{Set aside Cards}{set aside cards} once.
        Remove the card from the game after use.
      \end{itemize}\par
      \item \textbf{Change Starting Player} – Pass the Initiative Token to the other player. The odd rounds are started by the first player, and the even ones by the second player.\par
    \end{enumerate}\par
    \smallskip
  In the last Round the players don't draw Adventure cards, but they can still resolve a saved Adventure card. After the last round, the Adventure Phase ends and the final Combat starts.

  \subsection*{Start final Combat}
  Both players regain all used \svg{expert}.
  Then, they pick up their set-aside Deck, shuffle it, and draw cards up to their Hand Limit (depending on their current Hero's Level).
  They remove all \svg{paralysis} from recruited units, take miniatures of all of them and start the Combat on the Battlefield Board following \pagelink{Battlefield Combat}{their general rules}.
  The limit of 5 units doesn't apply here.
  \end{multicols}
\end{expansion}

\newpage
\begin{expansion}{battlefield}
  \begin{multicols}{2}
  \subsection*{Resolving Adventure Cards}
  There are two different types of Adventure cards: Combats or Events.
  \begin{multicols}{2}
    \footnotesize
    \begin{center}
      \includegraphics[width=\linewidth]{\cards/adventure_event.png}
      \imagecaption{Adventure-Event card}
      \begin{enumerate}
        \item[\textbf{1.}] Name
        \item[\textbf{2.}] Type
        \item[\textbf{3.}] Effect
      \end{enumerate}
    \end{center}
    \columnbreak

    \begin{center}
      \includegraphics[width=\linewidth]{\cards/adventure_combat.png}
      \imagecaption{Adventure-Combat card}
      \begin{enumerate}
        \item[\textbf{4.}] Combat Power Levels
        \item[\textbf{5.}] Battle Reward
        \item[\textbf{6.}] Basic Reward
      \end{enumerate}
    \end{center}
  \end{multicols}
  To resolve an \textbf{Adventure-Event card} just resolve the printed effect.
  \smallskip

  To resolve a \textbf{Adventure-Combat card} follow these steps:
  \begin{enumerate}
    \item \textbf{Declare one of the three available Combat Power Levels:}\index{Combat Power}
    You want to reach at least that power value to gain the full reward.
    \item \textbf{Determine your Combat Power:}
    Select 2 of your units and roll 1 Attack Die for each.
    If you have less than two units, select as many as you have.
    Sum all unit Attack values and all Attack Die results to determine your final Combat Power.
    \item \textbf{Get the Reward:}
    If your Combat Power is equal or higher than your declared Combat Power Level, take both the Basic and the Battle Reward.
    Otherwise take only the Basic Reward and put \svg{paralysis} at one of your units that took part in this Combat.
    This unit is now stunned.
    If you choose a stunned unit for a future Combat, it does not add its Attack value to your Combat Power.
    Then remove the \svg{paralysis} Token after Combat.
  \end{enumerate}\par

   After resolving a Adventure card store it in your own discard pile.
   E.~g.~the Adventure-Event card ``Obelisk'' grows stronger with every \textbf{own} Obelisk card played before.\par

  \subsection*{Additional Rules}
  \begin{itemize}
    \item Your limit for expert effects \svg{expert} applies for the entire Adventure Phase.
    You regain all used \svg{expert} at the beginning of the Combat phase.
    \item Your Deck isn't available during the Adventure Phase.
    All cards of your starting Deck are placed face down near your Hero Board and wait for Combat Phase.
    You can't use your Deck during adventure Phase.
    \item Each time an effect instructs you to \pagetarget{Set aside Cards}{gain new cards} (usually by leveling up or by resolving Adventure cards), you must proceed as follows:
    \begin{itemize}
      \item Cards, that are \textbf{useful only in Combat}, go directly in your Deck and wait there for Combat Phase.
      You can't use these cards during Adventure Phase.
      Spells go always into your Deck, cause their alternative effect (+1 \svg{empower}) is always useful in combat.
      \item All cards, that \textbf{aren't useful in Combat}, because they provide Resources (e.~g.~``Estates'') or \svg{movement}, or they give discount (e.~g.~``Legs of Legion'', ``Wisdom''), are \textbf{set aside} next to your Hero Board.
      You can use each of these cards only once.
      Remove them from play after using.
      They don't go in your Deck.
      \item If you are not sure how to handle a drawn card, be aware that there are not clear rules for every individual case and be prepared to apply house rules.
    \end{itemize}
    \item To resolve Morale effects use \pagelink{Morale Cards}{Morale cards} instead of Tokens and keep the \pagelink{Exception for Morale Cards}{exception rule} in mind.
  \end{itemize}
  \end{multicols}
\end{expansion}

\newpage
\begin{expansion}{battlefield}
    \subheader{Skirmish Mode}
    \begin{multicols}{2}
        \pagetarget{Skirmish}{The Skirmish Mode} has no adventure phase.
        You skip hat phase, prepare your Deck and you jump straight into the final Combat.\par
        \vspace*{1em}
        \subsection*{Deck Preparation}
        Both players decide on a game length: \textbf{short, medium or large}.
        Depending on the chosen game length both players modify their Hero's Level, their Deck (by drawing additional cards) and their Unit Deck as shown in the table below.\par

        \columnbreak
        \subsection*{Starting final Combat}
        After Modifying your Deck, shuffle it and draw cards up to your Hand Limit.
        The first player gets the Initiative Token.
        Then follow the rules for \pagelink{Battlefield Combat}{Combat on the Battlefield Board}.
        The limit of 5 units doesn't apply here.

        \vspace*{\fill}
    \end{multicols}

    \vspace*{1em}
    \hommtable{26}{
        \begin{tabularx}{0.98\linewidth}{>{\hsize=0.2\hsize\linewidth=\hsize}X
                >{\hsize=0.25\hsize\linewidth=\hsize}X
                >{\hsize=0.25\hsize\linewidth=\hsize}X
                >{\hsize=0.25\hsize\linewidth=\hsize}X
            }
            & \darkcell{\raisebox{-1.3\height}{Short Game}} & \darkcell{Medium Game} & \darkcell{\raisebox{-1.3\height}{Large Game}} \\
            \darkcell[1.2]{Hero Level} &
            \lightcell[1.2]{Level 3} &
            \lightcell[1.2]{Level 5} &
            \lightcell[1.2]{Level 7} \\
            \darkcell[1.2]{Specialty cards} &
            \lightcell[1.2]{Level I Specialty card} &
            \lightcell[1.2]{Level I and IV Specialty cards} &
            \lightcell[1.2]{All Specialty cards} \\
            \darkcell[1.8]{Ability cards} &
            \lightcell[1.8]{Draw 3 Ability cards and keep 2 of them.} &
            \lightcell[1.8]{Draw 4 Ability cards and keep 3 of them.} &
            \lightcell[1.8]{Draw 6 Ability cards and keep 4 of them.} \\
            \darkcell[1.8]{Artifact cards} &
            \lightcell[1.8]{Draw 2 Artifact cards and keep 1 of them.} &
            \lightcell[1.8]{Draw 3 Artifact cards and keep 2 of them.} &
            \lightcell[1.8]{Draw 4 Artifact cards and keep 3 of them.} \\
            \darkcell[2.2]{Spell cards} &
            \lightcell[2.2]{Draw 1 Spell card for a \svg[12]{might-yellow} Hero or 2 Spell cards for a \svg{magic-yellow} Hero.} &
            \lightcell[2.2]{Draw 2 Spell card for a \svg[12]{might-yellow} Hero or 4 Spell cards for a \svg{magic-yellow} Hero.} &
            \lightcell[2.2]{Draw 3 Spell card for a \svg[12]{might-yellow} Hero or 5 Spell cards for a \svg{magic-yellow} Hero.} \\
            \darkcell[2.2]{Units} &
            \lightcell[2.2]{Use all of your \svg{bronze} and \svg{silver} units on "Few" Site.} &
            \lightcell[2.2]{Use all of your \svg{bronze} and \svg{silver} units on "Pack" Site.} &
            \lightcell[2.2]{Use all of your units (\svg{bronze}, \svg{silver} and \svg{golden}) on "Pack" Site.} \\
        \end{tabularx}
    }
\end{expansion}
